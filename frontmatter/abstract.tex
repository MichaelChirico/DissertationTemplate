\begin{center}
    ABSTRACT\\
    \thetitle\\
    \vspace{.5in}
    \theauthor\\
    \theadvisor
\end{center}

\doublespaced
\noindent

This dissertation consists of three chapters on topics in public economics. The first chapter examines the labor market for public school teachers in Wisconsin. By stitching together publicly avaialable cross-sectional data to form a 20-year panel of teachers, I am able to replicate and extend the work of Hanushek, Kain and Rivkin who performed a similar analysis in Texas. The main takeaway is that teachers appear to select on wages, but that student characteristics appear more important in predicting teacher churn. In the second chapter, I present short-term analysis of a randomized-controlled trial designed to test the efficacy of active learning methods for teaching intermediate calculus to first-year college students. The results were inconclusive, suggesting substantial heterogeneity in student preferences and aptitudes for different styles of learning. The final chapter presents the analysis of a large-scale randomized-controlled trial evaluating the potential for messaging-based nudges to elicit increased real estate tax compliance in Philadelphia. Our primary conclusions are that most proposed messaging strategies are indistinguishable from a plainly-worded reminder bill (the exception being consequentialist letters threatening repercussive action absent compliance), but that the saliency \textit{per se} of a plainly-worded bill can induce late payers to remunerate more quickly.
