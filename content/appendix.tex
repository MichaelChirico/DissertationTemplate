\begin{appendices}
    \addtocontents{toc}{\protect\setcounter{tocdepth}{1}}
    \makeatletter
    \addtocontents{toc}{%
        \begingroup
        \let\protect\l@chapter\protect\l@section
        \let\protect\l@section\protect\l@subsection
    }

\chapter{Appendix to Chapter 1}


DPI WISEstaff data does not include a longitudinal identifier for
teachers, so we resorted to an alternative approach to matching teachers
from year to year. The essence of this algorithm relies on the inclusion
of four fields in the DPI data -- \texttt{first\_name},
\texttt{last\_name}, \texttt{nee} (former last name) and
\texttt{birth\_year}. By matching teacher using these identifiers, it is
possible to construct with high accuracy a panel of teachers from simple
teacher cross-sections\footnote{The code for this process was done using
  R and especially helped by the \texttt{data.table} package (Dowle and
  Srinivasan \citeyear{dowle}). The code to reproduce
  this entire project can be found at
  \url{https://github.com/MichaelChirico/wisconsin_teachers}. The script
  for the algorithm is \texttt{teacher\_match\_and\_clean.R}. The
  \texttt{README} file gives steps for full reproduction, including
  retrieving the raw data.}.

\subsection{Step 0: Pre-Processing}\label{step-0-pre-processing}

Prior to beginning the matching process, a number of steps are taken to
improve the quality of the raw data. The first is to incorporate as many
of the errata mentioned in Public Instruction
(\citeyear{dpi_errata})
as possible. All name variables are then converted to lower case, after
which we extract maiden names (identified for those missing a
DPI-supplied entry in \texttt{nee} as the part of the
\texttt{last\_name} field that appears in between parentheses or
surrounding a hyphen or forward slash). Generally, it appears the maiden
name comes in the data \emph{before} the hyphen, but we create the
\texttt{nee2} field to identify potential matches to the post-hyphen
name as well. A search was done of the data for irregular characters
(punctuation or numbers) which allowed several obvious typos to be
resolved (e.g., \texttt{l0is\ a\ dewey} was easy to resolve as being
\texttt{lois\ a\ dewey}), and this is implemented next.

Next, we create a ``clean'' version of the name fields which strips away
all whitespace, initials (lone letters), and punctuation. At this stage,
all observations which identically match another in the same year from
the viewpoint of the algorithm -- namely, those that match exactly
another observation on the cleaned first and last names and year of
birth -- are removed from the data since it would be impossible to tell
such teachers apart. A more ambitious treatment would attempt to use
other cues found in the duplicated records (ethnicity, subject/position
cues, etc.) to separate such teachers, but the marginal cost of doing so
was found to exceed the potential benefit considerably for the exercise
at hand (recall that only 0.7\% of total observations are lost in this
fashion). Finally, teachers in the first year of data (1994-95) are
assigned an ID starting from 1 using the within-year identifier provided
by DPI.

\subsection{Steps 1-21: Matching}\label{steps-1-21-matching}

The algorithm proceeds by iterating over years of the data. In each year
\(Y\), matches are found serially by progressively adjusting the
criteria for considering two observations to be from the same teacher as
follows:

\begin{enumerate}
\def\labelenumi{\arabic{enumi}.}
\tightlist
\item
  Match anyone who stayed in the same school -- i.e., match any teacher
  found in a year \(Y'<Y\) with the same \texttt{first\_name},
  \texttt{last\_name}, and \texttt{birth\_year} at the same
  \texttt{district} and \texttt{school}.
\item
  Find within-district switchers -- those who match on all but the
  \texttt{school} field from Step 1.
\item
  Find out-of-district switchers -- those who match on all but the
  \texttt{district} field from Step 2.
\item
  Find teachers that appear to have been married, but have not moved --
  their \texttt{nee} field in \(Y\) is matched to the
  \texttt{last\_name} fields in \(Y'<Y\), but otherwise the fields from
  Step 1 are all matched. We create a flag for teachers matched in this
  fashion called \texttt{married}.
\item
  Repeat Step 2 for those who appear to have married.
\item
  Repeat Step 3 for those who appear to have married. 7-9. Repeat Steps
  4-6 using the \texttt{nee2} field instead of \texttt{nee}. 10-18.
  Repeat Steps 1-9 using the ``cleaned'' version of the first name field
  that had non-alphabetic characters removed,
  \texttt{first\_name\_clean}. We create a flag for teachers matched in
  this fashion called \texttt{mismatch\_inits}.
\item
  Match individuals in the same school assigned to the same position
  (identified in \texttt{position\_code}) but with different years of
  birth to overcome potential noise in year of birth (most commonly, the
  year of birth is missing in some years). We create a flag for teachers
  matched in this fashion called \texttt{mismatch\_yob}.
\item
  Match individuals in the same \emph{district} assigned to the same
  position but with different years of birth. We do not extend this
  logic to find district switchers since the potential for erroneous
  match assignment in such a case is too great, and we neglect to extend
  the algorithm to use other cues from the data to facilitate matching
  in such cases.
\item
  Assign new IDs to all teachers in \(Y\) not matched in the first 20
  steps, incrementing from the highest ID recorded thus far.
\end{enumerate}

To help ensure we are matching to the most important observation of each
teacher, matching is always done to a teacher's highest-FTE observation
within a year (particularly important for Steps 19-20). Further, it is
sometimes the case that a given tuple of search keys matches more than
one teacher in the prior data; if so, these rows are simply ignored for
that step and such a teacher will go unmatched unless they are uniquely
pinned down in a subsequent step.

\begin{figure}[htbp]
\centering
\includegraphics{figures/match_step-1.pdf}
\caption{\label{fig:step}Frequency of Matching by Step}
\end{figure}

Figure \ref{fig:step} Shows that teachers are most commonly matched in
the first three steps, meaning data fidelity issues are not \emph{per
se} devastating. The real benefits of the algorithm are in the
subsequent steps, as a result of which an additional 33,538 teachers are
matched than would have been if only the original first and last name
fields as found in the raw data were used.

\subsection{Step 22: Post-Processing}\label{step-22-post-processing}

The panels implied by the matched IDs created by the algorithm still
have a substantial amount of data quality issues which we can only
address once teachers' multiple observations are associated, mainly
having to do with instability in certain observable characteristics
which should be constant over time. First, we cascade forward maiden
names (if a teacher has non-missing \texttt{nee} in a period \(Y\) and
it becomes missing in \(Y'>Y\), we replace it in \(Y'\) with its value
in \(Y\)); the same is done for the certification field,
\texttt{highest\_degree} (just as a teacher cannot erase marriage from
their past, so can they not make a degree disappear).

Next, we correct instability in the \texttt{ethnicity} (and
\texttt{gender}) fields when possible according to three steps: 1) it is
sometimes missing, in which case we simply overwrite it with the other
values found for that teacher; 2) at least 70\% of a teacher's
observations use the same ethnicity (or gender); or 3) there are at
least five people that share a last name with an ethnicity-ambiguous
teacher, at least 70\% of whom have one ethnicity (or gender), the idea
being that names like Xu or Gutierrez are strongly associated with a
particular ethnicity. This type of correction is uncommon enough not to
warrant an appeal to a more sophisticated approach commonly found in
natural language processing applications.

Lastly, we synergize the year of birth field for those matched in Steps
19 or 20 by assigning the one that appears most frequently for each
teacher; in the case of ties, we use a regression-to-the-mean-type logic
and assign the year which brings the teacher closer to the median age
observed in the data. More data-driven approaches (conditioning the
target median on the teacher's employer, position, year in the data,
etc.) were again eschewed for expediency.

\subsection{\texorpdfstring{Validity Check:
\texttt{file\_number}}{Validity Check: file\_number}}\label{validity-check-file_number}

Starting in the 2011-12 release, the DPI data begins to consistently
record a field called \texttt{file\_number} for teachers which generally
acts as a time-consistent ID (from verbiage gleaned from Public
Instruction
\citeyear{dpi_errata},
it appears this corresponds to a teaching license number). We looked for
instances of multiple file numbers and are content that the algorithm is
performing well -- only 78 teacher IDs were found to be associated with
more than one \texttt{file\_number}, with almost all of them having been
matched on Steps 1 - 3 (what should be the highest-accuracy steps).
Given a number of apparent transcription mistakes (i.e.,
\texttt{file\_number} differing by one digit in some years) and that the
\texttt{file\_number} does appear to change on occasion, even these 78
could be an overstatement of the number of incorrectly matched
individuals.


\chapter{Appendix to Chapter 3}


The appendix contains Tables \ref{balance2} and \ref{sh_lpm_mult}
which summarizes additional balance tests and robustness analyses
using all owners (including multiple property owners).Tables
\ref{sh_logit} and \ref{sh_logit_rob} report estimates based on Logit
models for unary owners and unary plus multiple owners.

\setcounter{table}{0}
\renewcommand{\thetable}{A\arabic{table}}


\begin{table}[htbp]
\caption{Robustness Analysis: Relative to Reminder (All Owners)}\label{sh_lpm_mult}
\begin{center}
\begin{tabular}{l c c c c }
\hline
 & \multicolumn{2}{c}{Ever Paid} & \multicolumn{2}{c}{Paid in Full} \\
 & One Month & Three Months & One Month & Three Months \\
Reminder     & $34.9$ & $56.5$ & $23.9$ & $41.8$ \\
\hline
Lien         & $4.8^{***}$  & $4.7^{***}$  & $3.3^{***}$  & $4.0^{***}$  \\
             & $(1.3)$      & $(1.3)$      & $(1.2)$      & $(1.3)$      \\
Sheriff      & $3.4^{***}$  & $4.6^{***}$  & $2.3^{**}$   & $3.6^{***}$  \\
             & $(1.3)$      & $(1.3)$      & $(1.2)$      & $(1.3)$      \\
Neighborhood & $-1.0$       & $-0.8$       & $-1.2$       & $-0.4$       \\
             & $(1.3)$      & $(1.3)$      & $(1.2)$      & $(1.3)$      \\
Community    & $-0.4$       & $-1.4$       & $-0.6$       & $-0.2$       \\
             & $(1.3)$      & $(1.3)$      & $(1.2)$      & $(1.3)$      \\
Peer         & $0.3$        & $-0.8$       & $0.4$        & $0.8$        \\
             & $(1.3)$      & $(1.3)$      & $(1.2)$      & $(1.3)$      \\
Duty         & $-1.3$       & $-0.2$       & $-1.0$       & $-0.8$       \\
             & $(1.3)$      & $(1.3)$      & $(1.2)$      & $(1.3)$      \\
\hline
Num. obs.    & 19333        & 19333        & 19333        & 19333        \\
\hline
\multicolumn{5}{l}{\scriptsize{$^{***}p<0.01$, $^{**}p<0.05$, $^*p<0.1$. Reminder values in levels; remaining figures relative to this}}
\end{tabular}
\end{center}
\end{table}

\begin{sidewaystable}[htbp]
\centering
\caption{Balance on Observables}
\label{balance2}
\begin{tabular}{lrrrrrrrc}
\hline
\multicolumn{9}{c}{Unary Owners} \\
  \hline
Variable & Reminder & Lien & Sheriff & Neighborhood & Community & Peer & Duty & $p$-value \\ 
   \hline
Amount Due (June) & \$1,256 & \$1,280 & \$1,315 & \$1,289 & \$1,290 & \$1,280 & \$1,299 & 0.98 \\ 
  Assessed Property Value & \$158,370 & \$130,642 & \$134,334 & \$159,079 & \$130,265 & \$130,936 & \$165,617 & 0.46 \\ 
  \# Owners & 2,419 & 2,429 & 2,416 & 2,387 & 2,441 & 2,416 & 2,432 & 0.99 \\ 
  \hline
\multicolumn{9}{c}{Unary and Multiple Owners} \\
  \hline
Variable & Reminder & Lien & Sheriff & Neighborhood & Community & Peer & Duty & $p$-value \\ 
   \hline
Amount Due (June) & \$1,593 & \$1,593 & \$1,590 & \$1,589 & \$1,583 & \$1,572 & \$1,583 & 1 \\ 
  Assessed Property Value & \$180,664 & \$155,499 & \$157,398 & \$180,172 & \$153,528 & \$155,438 & \$183,991 & 0.48 \\ 
  \% with Unary Owner & 87.6 & 88.0 & 87.5 & 86.4 & 88.4 & 87.5 & 88.1 & 0.42 \\ 
  \% Overlap with Holdout & 3.69 & 3.44 & 3.29 & 3.73 & 3.40 & 3.55 & 3.40 & 0.97 \\ 
  \# Properties per Owner & 1.27 & 1.26 & 1.26 & 1.32 & 1.26 & 1.26 & 1.26 & 0.67 \\ 
  \# Owners & 2,762 & 2,761 & 2,762 & 2,762 & 2,762 & 2,762 & 2,762 & 1 \\ 
  \hline
\multicolumn{9}{l}{\scriptsize{$p$-values in rows 1-5 are $F$-test $p$-values from regressing each variable on treatment dummies. A $\chi^2$ test was used for the count of owners.}} \\
\end{tabular}
\end{sidewaystable}


\begin{table}[htbp]
\caption{Short-Term Logistic Model Estimates (Unary Owners)}\label{sh_logit}
\begin{center}
\begin{adjustbox}{width=\textwidth,totalheight=0.9\textheight,keepaspectratio}
\begin{tabular}{l c c c c }
\hline
 & \multicolumn{2}{c}{Ever Paid} & \multicolumn{2}{c}{Paid in Full} \\
 & One Month & Three Months & One Month & Three Months \\
Holdout        & $-0.8$ & $0.1$       & $-1.2$ & $-0.4$ \\
\hline
Reminder        & $0.2^{***}$  & $0.2^{***}$ & $0.1^{*}$    & $0.1^{**}$   \\
               & $(0.1)$      & $(0.1)$     & $(0.1)$      & $(0.1)$      \\
Lien           & $0.4^{***}$  & $0.4^{***}$ & $0.3^{***}$  & $0.3^{***}$  \\
               & $(0.1)$      & $(0.1)$     & $(0.1)$      & $(0.1)$      \\
Sheriff        & $0.3^{***}$  & $0.4^{***}$ & $0.2^{***}$  & $0.3^{***}$  \\
               & $(0.1)$      & $(0.1)$     & $(0.1)$      & $(0.1)$      \\
Neighborhood   & $0.1$        & $0.1^{*}$   & $-0.0$       & $0.1$        \\
               & $(0.1)$      & $(0.1)$     & $(0.1)$      & $(0.1)$      \\
Community      & $0.2^{***}$  & $0.1^{*}$   & $0.1$        & $0.1^{*}$    \\
               & $(0.1)$      & $(0.1)$     & $(0.1)$      & $(0.1)$      \\
Peer           & $0.2^{***}$  & $0.1^{**}$  & $0.1$        & $0.1^{**}$   \\
               & $(0.1)$      & $(0.1)$     & $(0.1)$      & $(0.1)$      \\
Duty           & $0.1^{*}$    & $0.1^{**}$  & $0.0$        & $0.1$        \\
               & $(0.1)$      & $(0.1)$     & $(0.1)$      & $(0.1)$      \\
\hline
AIC            & 24493.1      & 26068.9     & 21605.6      & 26093.5      \\
BIC            & 24556.0      & 26131.7     & 21668.4      & 26156.3      \\
Log Likelihood & -12238.6     & -13026.4    & -10794.8     & -13038.7     \\
Deviance       & 24477.1      & 26052.9     & 21589.6      & 26077.5      \\
Num. obs.      & 19028        & 19028       & 19028        & 19028        \\
\hline
\multicolumn{5}{l}{\scriptsize{$^{***}p<0.01$, $^{**}p<0.05$,
    $^*p<0.1$. Holdout values in levels; remaining figures relative to
    this}}
\end{tabular}
\end{adjustbox}
\end{center}
\end{table}

\begin{table}[htbp]
\caption{Logit Estimates Including Multiple Owners}
\label{sh_logit_rob}
\begin{center}
\begin{adjustbox}{width=\textwidth,totalheight=0.9\textheight,keepaspectratio}
\begin{tabular}{l c c c c }
\hline
 & \multicolumn{2}{c}{All Owners} & \multicolumn{2}{c}{Unary Owners} \\
 & One Month & Three Months & One Month & Three Months \\
\hline
Lien           & $0.21^{***}$ & $0.20^{***}$ & $0.23^{***}$ & $0.22^{***}$ \\
               & $(0.06)$     & $(0.05)$     & $(0.06)$     & $(0.06)$     \\
Sheriff        & $0.15^{**}$  & $0.19^{***}$ & $0.16^{**}$  & $0.20^{***}$ \\
               & $(0.06)$     & $(0.05)$     & $(0.06)$     & $(0.06)$     \\
Neighborhood   & $-0.05$      & $-0.03$      & $-0.09$      & $-0.05$      \\
               & $(0.06)$     & $(0.05)$     & $(0.06)$     & $(0.06)$     \\
Community      & $-0.02$      & $-0.06$      & $0.00$       & $-0.04$      \\
               & $(0.06)$     & $(0.05)$     & $(0.06)$     & $(0.06)$     \\
Peer           & $0.01$       & $-0.03$      & $0.01$       & $-0.02$      \\
               & $(0.06)$     & $(0.05)$     & $(0.06)$     & $(0.06)$     \\
Duty           & $-0.06$      & $-0.01$      & $-0.06$      & $-0.01$      \\
               & $(0.06)$     & $(0.05)$     & $(0.06)$     & $(0.06)$     \\
\hline
AIC            & 25179.24     & 26349.91     & 21922.44     & 23174.00     \\
BIC            & 25234.33     & 26405.00     & 21976.61     & 23228.16     \\
Log Likelihood & -12582.62    & -13167.95    & -10954.22    & -11580.00    \\
Deviance       & 25165.24     & 26335.91     & 21908.44     & 23160.00     \\
Num. obs.      & 19333        & 19333        & 16940        & 16940        \\
\hline
\multicolumn{5}{l}{\scriptsize{$^{***}p<0.001$, $^{**}p<0.05$, $^*p<0.1$}}
\end{tabular}
\end{adjustbox}
\end{center}
\end{table}


\end{appendices}
