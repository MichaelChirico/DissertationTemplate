\documentclass[12pt,]{article}
\usepackage{lmodern}
\usepackage{setspace}
\setstretch{1.5}
\usepackage{amssymb,amsmath}
\usepackage{ifxetex,ifluatex}
\usepackage{fixltx2e} % provides \textsubscript
\ifnum 0\ifxetex 1\fi\ifluatex 1\fi=0 % if pdftex
  \usepackage[T1]{fontenc}
  \usepackage[utf8]{inputenc}
\else % if luatex or xelatex
  \ifxetex
    \usepackage{mathspec}
  \else
    \usepackage{fontspec}
  \fi
  \defaultfontfeatures{Ligatures=TeX,Scale=MatchLowercase}
\fi
% use upquote if available, for straight quotes in verbatim environments
\IfFileExists{upquote.sty}{\usepackage{upquote}}{}
% use microtype if available
\IfFileExists{microtype.sty}{%
\usepackage{microtype}
\UseMicrotypeSet[protrusion]{basicmath} % disable protrusion for tt fonts
}{}
\usepackage[margin=1in]{geometry}
\usepackage{hyperref}
\hypersetup{unicode=true,
            pdftitle={Teacher Turnover in Wisconsin},
            pdfauthor={Michael Chirico},
            pdfborder={0 0 0},
            breaklinks=true}
\urlstyle{same}  % don't use monospace font for urls
\usepackage{graphicx,grffile}
\makeatletter
\def\maxwidth{\ifdim\Gin@nat@width>\linewidth\linewidth\else\Gin@nat@width\fi}
\def\maxheight{\ifdim\Gin@nat@height>\textheight\textheight\else\Gin@nat@height\fi}
\makeatother
% Scale images if necessary, so that they will not overflow the page
% margins by default, and it is still possible to overwrite the defaults
% using explicit options in \includegraphics[width, height, ...]{}
\setkeys{Gin}{width=\maxwidth,height=\maxheight,keepaspectratio}
\IfFileExists{parskip.sty}{%
\usepackage{parskip}
}{% else
\setlength{\parindent}{0pt}
\setlength{\parskip}{6pt plus 2pt minus 1pt}
}
\setlength{\emergencystretch}{3em}  % prevent overfull lines
\providecommand{\tightlist}{%
  \setlength{\itemsep}{0pt}\setlength{\parskip}{0pt}}
\setcounter{secnumdepth}{0}
% Redefines (sub)paragraphs to behave more like sections
\ifx\paragraph\undefined\else
\let\oldparagraph\paragraph
\renewcommand{\paragraph}[1]{\oldparagraph{#1}\mbox{}}
\fi
\ifx\subparagraph\undefined\else
\let\oldsubparagraph\subparagraph
\renewcommand{\subparagraph}[1]{\oldsubparagraph{#1}\mbox{}}
\fi

%%% Use protect on footnotes to avoid problems with footnotes in titles
\let\rmarkdownfootnote\footnote%
\def\footnote{\protect\rmarkdownfootnote}

%%% Change title format to be more compact
\usepackage{titling}

% Create subtitle command for use in maketitle
\newcommand{\subtitle}[1]{
  \posttitle{
    \begin{center}\large#1\end{center}
    }
}

\setlength{\droptitle}{-2em}
  \title{Teacher Turnover in Wisconsin}
  \pretitle{\vspace{\droptitle}\centering\huge}
  \posttitle{\par}
  \author{Michael Chirico}
  \preauthor{\centering\large\emph}
  \postauthor{\par}
  \predate{\centering\large\emph}
  \postdate{\par}
  \date{May 02, 2017}

\usepackage{theapa}
\usepackage{array}
\usepackage{multirow}
\newcolumntype{R}[1]{>{\raggedleft\arraybackslash}p{#1\linewidth}}
\usepackage{rotating}

\begin{document}
\maketitle
\begin{abstract}
Given the consistently-affirmed importance of teacher quality to student
success, understanding teacher churn is crucial to formulating and
evaluating teacher labor market policy. This paper replicates and
extends the analysis of Hanushek, Kain, and Rivkin
(\protect\hyperlink{ref-hanushek}{2004}) over a longer and more recent
time period in Wisconsin and confirms all of its major findings, namely
that while inter-district pay differentials are a significant
determinant of turnover, school quality measures are much better
predictors of all three types of churn -- within and between school
districts and out of local public schools.
\end{abstract}

\section{Introduction}\label{introduction}

Good teachers have large impacts on student achievement\footnote{See,
  e.g., Rockoff (\protect\hyperlink{ref-rockoff}{2004}).}. It is
therefore imperative for public schools to be able to attract and retain
high-quality teachers. Of preeminent concern for policymakers, then, is
the strength of the various manipulable levers at their disposal for
influencing teacher labor markets. More specifically, state education
administrators would like to identify the policy implications of various
tools on three types of teacher mobility: intra-district switching,
where due to the collectively bargained nature of most teachers'
salaries, only nonpecuniary considerations matter, inter-district
switching, where teachers move to another school district in the same
state, and exo-district switching, where teachers leave the public
teaching workforce entirely\footnote{Policies that affect the supply and
  quality of new teachers to the profession may also be of considerable
  importance to replenishing and improving the stock of teachers over
  time, but we do not consider these channels in this work. See Harris
  and Sass (\protect\hyperlink{ref-harris}{2011}), Wayne et al.
  (\protect\hyperlink{ref-wayne}{2008}), and Boyd et al.
  (\protect\hyperlink{ref-boyd2009}{2009}).}.

In this paper, we consider pecuniary and non-pecuniary predictors of
various types of teacher churn in replicating the analyses of Hanushek,
Kain, and Rivkin (\protect\hyperlink{ref-hanushek}{2004}) in a new
context (Wisconsin) and time horizon (2000 - 2007). The headline results
of Hanushek, Kain, and Rivkin (\protect\hyperlink{ref-hanushek}{2004})
were ``that teacher mobility is much more strongly related to
characteristics of the students, particularly race and achievement, than
to salary, although salary exerts a modest impact once compensating
differentials are taken into account.'' We confirm the pith of this
conclusion, namely that student characteristics are a much better
predictor of turnover than are wage differentials, though we come to
different conclusions regarding more specific points. In fact, while we
do find strong evidence that the socioeconomic makeup of a teacher's
district predicts turnover (and that there is heterogeneity in this
effect by teacher race), the evidence we find for the importance of
wages and student achievement is far from compelling.

We explore to the extent possible potential contributors to this
discrepancy in results; most salient are the differences between Texas,
where Hanushek, Kain, and Rivkin
(\protect\hyperlink{ref-hanushek}{2004}) conduct their study, and
Wisconsin. Wisconsin is a largely rural state -- its largest
city/metropolitan area (in fact it is the only city considered to be
``large'' for NCES reporting purposes), Milwaukee, currently has roughly
600,000 residents (1,500,000 including the metropolitan area), making it
around the 30th-largest city in the United States. By contrast, Texas
has six cities larger than this, with El Paso being the nearest in size
to Milwaukee. Though the non-urban parts of Texas are themselves
sparsely populated and distinctly rural, the more uniform lack of major
population centers in Wisconsin is likely to be reflected in
considerably different preferences among local residents for various
aspects of potential teaching positions.

\section{Literature Review}\label{literature-review}

Because the potential policy implications of turnover in the teaching
profession (from human capital and equity/distributional perspectives
both) are far-reaching and polypartisan, the literature on
turnover-related topics in education is extensive. As relates to this
paper, there are five broad (and often overlapping) categories of
inquiry: the relationship between turnover and wages, which has tended
to focus on ``opportunity wages'' outside of the field of education; the
relationship between turnover, school demographics, and other
nonpecuniary benefits, which has tended to focus on distributional
inequalities--whether teachers with certain characteristics are more or
less likely to be teaching certain disadvantaged groups; the
relationship between turnover and teacher quality as measured by student
performance, usually value added (VA); collective bargaining agreements
in education, focusing by and large on the implications (or lack
thereof) of seniority-preferential clauses; and the recent phenomenon of
specific retention incentives, the provisioning of wage bonuses to
teachers willing to teach in high-needs schools.

One of the earliest papers attempting to rigorously investigate turnover
was a panel study of teachers in Michigan by Murnane and Olsen
(\protect\hyperlink{ref-murnane}{1990}), who used college degree field
wages outside of education as opportunity wages, finding the expected
lower exit rate for teachers with higher wages in teaching relative to
the authors' defined alternative. Dolton and Van der Klaauw
(\protect\hyperlink{ref-dolton}{1999}) use panel data on university
graduates in the United Kingdom to estimate a competing risks model of
the decision to leave teaching entirely, finding results in line with
Murnane and Olsen (\protect\hyperlink{ref-murnane}{1990}). Returning to
panel studies in the US, Loeb and Page
(\protect\hyperlink{ref-loeb}{2000}) use PUMS data to get an idea of
teacher relative wages in many states and find that dropout rates fall
when teacher relative wages are high. Stinebrickner
(\protect\hyperlink{ref-stinebrickner}{2002}) also uses panel data (this
time NLS-72) to track both teachers and non-teachers, focusing in
particular on young teachers who leave the profession for long stints,
and finds that the best predictor of female exit is recent childbearing,
which is an important consideration for all work related to teacher
turnover because such a high percentage (76 nationwide) of teachers are
female. Lastly, Hanushek, Kain, and Rivkin
(\protect\hyperlink{ref-hanushek}{2004}) focuses on teachers in Texas
and emphasizes that the characteristics of students are much stronger
factors in predicting teacher exit than are wages (while also affirming
the statistical significance of pay).

While wages have been found consistently to have some measurable effect
on teacher turnover, it is impossible to explain within-district
migration (which constitutes a large portion of switching--as much as
50\%) through wage-only channels because contracts are fixed at the
district level. As such, another strand of literature has chosen to
focus on the nonpecuniary aspects of the decision to take a teaching
job--school environment/rapport, student enthusiasm, neighborhood
characteristics, etc.--usually by directing attention to a single
district so that any wage-based considerations are stifled, as is the
case for Boyd et al. (\protect\hyperlink{ref-boyd2005}{2005}) and Engel,
Jacob, and Curran (\protect\hyperlink{ref-engel}{2014}). Boyd et al.
(\protect\hyperlink{ref-boyd2005}{2005}) track early-career teachers in
New York City as they quit or transfer out of the city, and most
importantly finds that commuting time is an important, often overlooked
aspect of location preference. Engel, Jacob, and Curran
(\protect\hyperlink{ref-engel}{2014}) leverages a unique data set from
Chicago Public School job fairs which affords them a rather strong
measure of teachers' demand for vacancies, neutralizing the influence of
school administration's behavior on turnover (through poor match
selection or other means). The authors contribute evidence that the
school's neighborhood (perhaps due to ambient crime or other
reputational effects good and bad) is a better predictor of teachers'
preference than distance from home, going somewhat against the grain of
Boyd et al. (\protect\hyperlink{ref-boyd2005}{2005}). Scafidi, Sjoquist,
and Stinebrickner (\protect\hyperlink{ref-scafidi}{2007}) examine
statewide data from Georgia, but ignore wage effects, choosing instead
to focus on disentangling the contributions of low student achievement
and minority status to turnover; they find that minority status is the
more salient associate of teacher exit.

The key element missing from all of the above studies is perhaps the
most important consideration in the issue of teacher turnover--teacher
quality. None of the studies above have student-teacher matched data,
and so are unable to directly associate student outcomes with any given
teacher. If, with respect to any measure of quality you would like, we
find that transitioning teachers are identical to their replacements,
the issue of teacher turnover is not, in fact, much of an issue. Thus,
the recent trend in the literature to incorporate measures of teacher
quality (in large part made possible by a trend towards administrative
records allowing students to be linked to teachers and tracked over
time) in considerations of teacher turnover has made big strides in
addressing the most policy-relevant questions to be asked. The most
common and widely accepted measure of teacher quality is VA\footnote{The
  most commonly cited expositions on value-added, its validity, and so
  on are probably Rivkin, Hanushek, and Kain
  (\protect\hyperlink{ref-rivkin}{2005}), an extensive exploration of
  the predictive powers of empirical Bayes VA measures; and Chetty,
  Friedman, and Rockoff
  (\protect\hyperlink{ref-chettyI}{2014}\protect\hyperlink{ref-chettyI}{a})
  and Chetty, Friedman, and Rockoff
  (\protect\hyperlink{ref-chettyII}{2014}\protect\hyperlink{ref-chettyII}{b}),
  the largest-scale study of long-term inferences based on VA.} (in its
various guises), and the literature has begun to incorporate such
measures into studies of teacher turnover. Hanushek and Rivkin
(\protect\hyperlink{ref-hanushek2010}{2010}) consider VA as a measure of
teacher productivity, and ask if common results of labor search theory
(namely that turnover falls with tenure and that turnover is negatively
associated with match-specific productivity) continue to hold in the
education labor market. In fact, the authors find that the teachers most
likely to switch schools are those with low measured match quality, and
especially that those who leave teaching entirely are those with the
lowest match quality. The results are more pronounced for schools with
high proportions of low-SES students, which has strong policy
implications, as it appears the best teachers in high needs schools are
the least likely to change jobs. Goldhaber, Gross, and Player
(\protect\hyperlink{ref-goldhaber2007}{2007}) performs a similar
analysis with the longitudinal data of North Carolina and comes to
similar conclusions, strengthening the robustness of the results.
Lastly, Goldhaber, Lavery, and Theobald
(\protect\hyperlink{ref-goldhaber2015}{2015}) examine the inequity in
the distribution of teacher quality by high-needs groups in Washington
state, and find that for all three measures of quality (teacher
experience, licensure exam score, and VA), the distribution of teachers
favors the less needy (as measured by free/reduced-price lunch status,
minority status, and low prior academic achievement).

The aforementioned papers have tended to keep the collective bargaining
aspect of salary determination for teachers out of the spotlight, if
largely for reasons of data restrictions. Nevertheless, it stands to
reason to believe that the rigid structure of union-negotiated contracts
could serve to contribute in a large way to teacher turnover. Ballou and
Podgursky (\protect\hyperlink{ref-ballou}{2002}) give much descriptive
evidence of the shape of the wage-tenure profile, rooted in a data set
collected by the Department of Defense and published by the AFT. They
find that seniority premia in education largely mirror those in more
traditional white collar professins, that steeper profiles are
associated with less turnover, and that district financial and
demographic conditions alone are insufficient to explain variation in
contracts. Another common (and recently quite controversial, as
evidenced by the contention in the ongoing contract negotiations in
Philadelphia) feature of union-negotiated teacher contracts are
seniority priviliges--preferential treatments granted to teachers in
voluntary and involuntary transfers. Moe
(\protect\hyperlink{ref-moe}{2006}) codes contracts from 158 districts
in California according to the strength of seniority rights therein
guaranteed to teachers and finds that such rights are associated with
the distribution of teachers across schools (measuring quality as
experience and certification) in a way that serves to harm minorities.
Revisiting California with a slightly different sample and definition of
the ``determinacy'' of the contracts with respect to seniority, Koski
and Horng (\protect\hyperlink{ref-koski}{2007}) come to the opposite
conclusion--that there is no such relationship. As a rebuttal, Anzia and
Moe (\protect\hyperlink{ref-anzia}{2014}) pin the difference in results
on the exclusion in Moe (\protect\hyperlink{ref-moe}{2006}) of small
school districts, where it appears that the entrenchment of bureaucracy
falters and the rigidity of contract language wane, a claim which they
support by repeating their analysis with the inclusion of an interaction
for district size--indeed, for small districts the result of Koski and
Horng (\protect\hyperlink{ref-koski}{2007}) holds, while the insight of
Moe (\protect\hyperlink{ref-moe}{2006}) holds in larger districts.
Cohen-Vogel, Feng, and Osborne-Lampkin
(\protect\hyperlink{ref-cohenvogel}{2013}) use data from Florida and
their results align with those of Koski and Horng
(\protect\hyperlink{ref-koski}{2007}) (though they neglect to nuance
their results by district size).

Finally, an emerging strand of literature is looking at the potential
for transfer bonuses and retention incentives to positively affect
student outcomes. Fulbeck (\protect\hyperlink{ref-fulbeck}{2014})
analyzes a scheme in place in Denver whereby teachers who choose to
transfer to high-needs schools (low-performing) are given recurring
bonus pay, and those initially stationed there are given retention
incentives. She concludes that recipients of incentives are
significantly less likely to switch jobs, as driven by a reduction in
district exit rates and especially by teachers whos incentive payments
exceed \$5,000. Glazerman et al.
(\protect\hyperlink{ref-glazerman}{2013}) evaluate the Talent Transfer
Initiative, a randomized controlled trial conducted in 10 districts
whereby high-performance teachers were given \$20,000 over the course of
two years as reward for transferring the identified high-needs schools,
and conclude that there were significant effects on teacher retention as
well as on student outcomes. Two highly germane papers investigate the
impact in Wisconsin on teachers of Governor Scott Walker's flagship
policy, Act 10, which severely limited the scope for collective
bargaining in the state. Litten (\protect\hyperlink{ref-litten}{2016})
uses differences in contract renewal dates surrounding the policy's
enactment to evince the effect of unionization on teachers' wages, and
finds the lack of union bargaining power reduced teacher compensation by
8\%. Biasi (\protect\hyperlink{ref-biasi}{2017}) constructs value-added
measures from grade-level test results and concludes that the move to
individually-negotiated salaries in some districts had a significant
impact on teacher quality and student outcomes in such districts, while
also cautioning that most of these gains are competition-based, so that
scaling up the system state-wide would have an impact limited to a boost
from the exit of low-quality teachers.

\section{Data}\label{data}

The State of Wisconsin's Department of Public Instruction (DPI) releases
annual Salary, Position \& Demographic reports through the WISEstaff
data collection system, and these reports represent ``a point-in-time
collection of all staff members in public schools as of the 3rd Friday
of September\ldots{}'' (Public Instruction
\protect\hyperlink{ref-dpi}{2017}), which serve as the primary source of
data on teachers in this paper. Data are available at the
position-teacher level cross-sectionally, with each entry corresponding
to one of possibly several positions held by each school district
employee. Identifiers in each file permit unique identification of an
employee within a given year, but this identifier does not follow
teachers between years. To overcome this substantial hurdle to
identifying teacher mobility, data are first fed through the matching
algorithm described in further detail in the appendix. Essentially, we
are aided by the presence of various imperfect identifiers which are
more stable over time, most crucially teachers' first and last names and
years of birth. By building on these covariates and incorporating some
limited fuzzy matching techniques, we construct a panel of teachers
spanning the 1994-95 academic year (AY) through AY2015-16.\footnote{For
  brevity, we herein refer to academic years by the spring year, e.g.,
  AY2003-04 will be simply 2004.}

As noted in the companion paper, the introduction of Wisconsin Act 10
introduced a substantial structural break in the labor market for
Wisconsin teachers, so we include only data from 2000-2010 to avoid
conflating the effects of this policy on teacher turnover, a topic
covered in more detail in the companion paper and elsewhere, with the
earlier functioning of the labor market (i.e., we do not want to mix the
results from distinct equilibria of the teacher labor market, but would
prefer to analyze the pre- and post-Act-10 markets separately). We drop
all employees who are not full-time, full-year regular teachers of a
major core subject (all-purpose elementary teachers or English/Math) at
a single regular public school with a Bachelor's or Master's degree and
fewer than 50 years' recorded experience; taken together, these
restrictions eliminate 79\% of employees, the lion's share of which come
from eliminating substitutes/support staff and teachers non-core
subjects\footnote{We also eliminate any teacher who appears in any role
  besides ``Teacher'' in any year. In particular, this eliminates a
  nontrivial number of educators who either begin their career with an
  ``ease-in'' period or end it with a ``soft retirement'' period, during
  which they act as a substitute teacher before or after a career
  otherwise focused on teaching. Such teachers often have part-time
  roles at several local schools, which introduces sufficient ambiguity
  in the definition of mobility so as to obscure interpretation of
  results, so we opt for a stricter definition of full-time teaching
  than is completely necessary.}. We then eliminate teachers with
missing information on their subsequent school or district and teachers
with instability in their recorded ethnicity, as well as teachers not
categorized as white, black, or Hispanic, eliminating a further 0.2\% of
all employees\footnote{Wisconsin teachers are predominantly white
  (96\%).}. Finally, we drop teachers' multiple positions by keeping
only the highest-intensity position for each teacher, as measured by
full-time equivalency, resulting in a final count of 287,811
teacher-year observations.

This data is also used for the incorporation of counterfactual salary
calculations, by incorporating the salary schedules estimated in the
companion paper. Details of the fit procedure can be found there, but
essentially salary schedules are computed as monotonicity- and
concavity-constrained median-targeted splines (Ng and Maechler
\protect\hyperlink{ref-ng}{2007}) for each level of certification
(Bachelor's or Master's degree) in each district in each year\footnote{One
  difference is that the salaries included in the payscales estimation
  were less restrictive with respect to included subject areas. This was
  done since contracts are collectively bargained at the district level
  for all teachers, with scant mention of subject area in wage
  determination.}. Data sparsity led this procedure to be unreliable in
many cases, so ultimately around 28\% of teachers have missing salary
information\footnote{More specifically, we eliminate district-years
  featuring less than 20 teachers, less than 7 distinct levels of
  observed experience, or less than 5 unique values of the two measures
  of pay (salary and fringe benefits) in either degree track. HKR
  include like-minded restrictions, but combine teachers of different
  certification within an experience level.}, mostly in rural districts
or other districts with only one or two schools and a small number of
students.

We supplement the DPI teacher salary data set in several ways to
incorporate data about other characteristics of schools and districts in
Wisconsin. To get school- and district-level measures of socioeconomic
makeup (percentage of students who are black or Hispanic or eligible for
free/reduced lunches) and community type/urbanicity, we tap the Universe
Surveys from the National Center for Education Statistics' Common Core
of Data, which provide this information on a yearly basis for all years
in the study\footnote{The method of recording urbanicity by the Common
  Core switched from being ``metropolitan-centric'' to being
  ``urban-centric'' for Wisconsin from 2006 (Sable
  \protect\hyperlink{ref-sable}{2009}). We map the codes corresponding
  codes to match those used by HKR as well as possible, and use the data
  file from 2006, which has both types of code for all US districts, to
  confirm that this correspondence is by and large working as intended.
  For a small number of districts/schools with missing urbanicity codes
  in certain years, we use information from other years to inform
  urbanicity.}. At the district level, we also use this data to compute
class size and the size of the student body.

Lastly, we turn to DPI's public data again to get school- and
district-level performance metrics. While Hanushek, Kain, and Rivkin
(\protect\hyperlink{ref-hanushek}{2004}) were able to obtain school- and
district-level average scale scores on a standardized test in Texas,
such a metric is not publicly available in Wisconsin for all years.
Instead, we calculate student proficiency rates for each school and
district as the percentage of test-takers deemed to be at grade level in
mathematics or reading in a given year on the Wisconsin Knowledge and
Concepts Examination (WKCE), which is administered to 4th, 8th, and
10th-grade students.

\section{Results}\label{results}

\begin{table}[htbp]
\centering
\begin{tabular}{p{.12\linewidth}p{.14\linewidth}p{.17\linewidth}p{.10\linewidth}p{.17\linewidth}R{.13}}
  \hline
 & \multicolumn{4}{c}{Percent of Teachers Who} & \\ \cline{2-5}
Teacher Experience & Remain in Same School & Change Schools Within District & Switch Districts & Exit Wisconsin \mbox{Public Schools} & Number of Teachers \\ 
  \hline
1-3 years & 79.7 & 7.1 & 6.0 & 7.2 & 41,042 \\ 
  4-6 years & 86.6 & 5.6 & 3.3 & 4.5 & 37,770 \\ 
  7-11 years & 90.6 & 5.0 & 1.8 & 2.6 & 54,623 \\ 
  12-30 years & 92.4 & 4.1 & 0.6 & 2.9 & 129,002 \\ 
  >30 years & 78.5 & 6.7 & 1.1 & 13.7 & 25,374 \\ 
  All & 88.3 & 5.1 & 2.0 & 4.6 & 287,811 \\ 
   \hline
\end{tabular}
\caption{Year-to-year Transitions of Teachers by Experience, 2000-10} 
\label{tbl:move_by_exp}
\end{table}

Table \ref{tbl:move_by_exp} replicates Table 1 of Hanushek, Kain, and
Rivkin (\protect\hyperlink{ref-hanushek}{2004}) (HKR)\footnote{This and
  subsequent analyses were greatly facilitated by several facilities of
  the R programming language, for which due credit must be given to R
  Core Team (\protect\hyperlink{ref-r}{2016}), RStudio Team
  (\protect\hyperlink{ref-rstudio}{2017}), Xie
  (\protect\hyperlink{ref-xie}{2016}), Leifeld
  (\protect\hyperlink{ref-leifeld}{2013}), Dahl
  (\protect\hyperlink{ref-dahl}{2009}), Henningsen and Toomet
  (\protect\hyperlink{ref-henningsen}{2011}), Zeileis and Hothorn
  (\protect\hyperlink{ref-zeileis2002}{2002}), Zeileis
  (\protect\hyperlink{ref-zeileis2004}{2004}), Zeileis
  (\protect\hyperlink{ref-zeileis2006}{2006}) and Croissant
  (\protect\hyperlink{ref-croissant}{2012}).}, and as HKR found in
Texas, most turnover in Wisconsin is happening within districts and out
of the profession. In Wisconsin, the fraction of teachers transitioning
among districts is vanishingly small after a ``burn-in'' period of
roughly 6 years -- only 1\% of such teachers do so (compared with 3.1\%
for the comparable group in HKR), but is still relatively higher among
the youngest teachers -- roughly twice as high for the ``probationary''
teachers (1-3 years' experience) as for teachers with 7-11 years'
experience in both states.

By contrast, movement patterns within districts in the two states are
very similar, lending weight to teachers ``earning their stripes''
within a district to be able to choose the best schools as a privilege
of seniority. As expected, we also observe a U-shaped pattern in
teachers exiting Wisconsin public schools, which jives with two types of
quits. Early-career quitters who change to private schools, change state
of residence, or change professions, and late-career quitters who retire
-- this is epecially evident among teachers with more than 30 years'
experience, a group which sees a mass exodus of fully 15 percent of its
teachers annually. Results not included here break down the exit rates
by experience level, where this dichotomy is even more dramatic --
first-year exit rates are about 8 percent and quickly level off at
around 2 percent before spiking again past around 25 years.

As examined further below, the low rate of switches between districts
appears to be owing to the generally more rural nature of Wisconsin
vis-à-vis. Texas. To wit, Milwaukee is the only major urban area in the
state, and its population (2010 Census) of 594,833 would rank 7th in
Texas. This means that two major types of movers in the HKR data --
Large Urban - Large Urban and Suburban - Large Urban -- are limited
within the state to ending up in a relatively minor metropolitan area.
HKR don't provide any results disaggregated by city, precluding any
attempts to compare these numbers more comparably to those that would
obtain from eliminating the largest cities in Texas.

\begin{sidewaystable}[htbp]
\centering
\begin{tabular}{lrrrrrrr}
  \hline
 & \multicolumn{4}{c}{\multirow{2}{*}{Percent of Teachers Who Move to}} & \multirow{4}{*}{\parbox{0.09\linewidth}{Number Teachers Changing Districts}} & \multirow{4}{*}{\parbox{0.07\linewidth}{Percent of Origin Teachers}} & \multirow{4}{*}{\parbox{0.09\linewidth}{Change in Share of Teachers 2000-06}}\\
 & \multicolumn{4}{c}{} & & & \\ \cline{2-5}
& & & & & & & \\
Origin Community & Large Urban & Small Urban & Suburban & Rural &  &  &  \\ 
  \hline
I. All teachers & & & & & & & \\
\quad Large Urban & 17.4 & 15.8 & 48.7 & 18.1 & 820 & 2.7 & 0.3\% \\ 
  \quad Small Urban & 3.7 & 13.4 & 44.7 & 38.2 & 642 & 1.2 & 0.0\% \\ 
  \quad Suburban & 3.5 & 16.0 & 44.1 & 36.4 & 1,428 & 1.9 & 3.7\% \\ 
  \quad Rural & 0.7 & 11.2 & 24.2 & 64.0 & 2,837 & 2.2 & -4.0\% \\ 
\multicolumn{3}{l}{II. Probationary teachers (1-3 years experience)} & & & & & \\
  \quad Large Urban & 15.8 & 17.9 & 47.6 & 18.8 & 437 & 5.0 &  \\ 
  \quad Small Urban & 5.1 & 14.0 & 46.2 & 34.8 & 271 & 3.9 &  \\ 
  \quad Suburban & 4.3 & 16.2 & 41.0 & 38.4 & 561 & 5.5 &  \\ 
  \quad Rural & 0.3 & 10.9 & 24.4 & 64.5 & 1,204 & 8.0 &  \\ 
   \hline
\end{tabular}
\caption{Destination Community Type for Teachers Changing Districts, by Origin Community Type and Teacher Experience Level} 
\label{tbl:markov}
\end{sidewaystable}

Table \ref{tbl:markov} replicates HKR Table 2, and supports its most
important conclusions. HKR argue that there is little support for the
idea that scores of young teachers are using large urban schools as a
training ground before ``settling down'' with easier assignments in the
suburb, based on the general low level of turnover from Large Urban
districts. We affirm the scarcity of transitions from districts in
Milwaukee, while also noting that such a path is certainly present, as
evidenced by the majority of those who do leave Large Urban districts
ending up in a Surburban district in both settings. HKR also observe
that the likelihoods of remaining in the same school and of quitting are
roughly the same for urban and suburban teachers, an observation which
we can confirm in Wisconsin. We further note that while Table
\ref{tbl:markov} only presents a cross-sectional picture, the
career-long trend reaffirms this -- only 3.2\% of teachers starting
their careers at a large urban district ever work at a suburban
district. Lastly, we echo the suggestion of HKR that this phenomenon
cannot be driven \emph{per se} by demand-side constraints -- in our time
period of observation, we observe only 1,462 urban teachers change
districts, whereas 3,211 teachers were hired in suburban districts,
though of course this does not rule out arguments based for example on
stricter screening of applicants transferring from urban districts.

We note, however, that though tales of flight from troubled urban
districts is apparently anecdotal, it is far from apocryphal. To wit,
while 50 percent of districts have a net inflow (arrivals less
departures) of four or fewer teachers (in absolute value), Milwaukee's
net outflow was 531 teachers, and the five highest-inflow districts, all
suburbs of Milwaukee or districts adjacent the main university town of
Madison, saw in total an inflow of 231 teachers in this time. This being
a two-sided market, this state of affairs is perhaps largely
attributable to the dynamic nature of student populations at these
districts -- but these, as well, are reflective of the appeal of the
districts to parents (and teachers as parents).

\begin{figure}[htbp]
\centering
\includegraphics{figures/tx_wi_urb_dist-1.pdf}
\caption{\label{fig:ti_wi_urb}Comparison of the Prevalence of Different
Community Types}
\end{figure}

As mentioned in the discussion of Table \ref{tbl:move_by_exp}, the major
difference with respect to quantities observed in Texas appears to be
driven in differences in the urban landscape between Texas and
Wisconsin\footnote{We also note a difference in the relative shift in
  population between the two states -- Texas observed dramatic changes
  in its community type distribution over the period of study of only 4
  years, while Wisconsin only saw some movement from Rural to Suburban
  communities.}. This is supported by the overall similarity of
magnitudes of transition rates to community types besides Large Urban in
the two papers. Figure \ref{fig:ti_wi_urb} depicts this difference in
landscape by comparing the distribution of community types in Texas and
Wisconsin (bar widths reflect the relative quantity of districts in
Texas and Wisconsin). While both states are majority-rural, the
non-rural part of Texas is comparatively urbanized, whereas more than
90\% of Wisconsin districts are non-urban.

Returning to Table \ref{tbl:move_by_exp}, we see that, as in HKR, the
``stickiest'' community type is Rural -- over 60\% of Rural teachers
remain Rural in both papers, and even fewer Rural Wisconsin teachers end
up in a big city than is the case for Texas. This may reflect the
similarity in prevalence of rural districts in the two states. Lastly,
we also find broad similarity in the community type transition patterns
of younger teachers as compared to all teachers.

\begin{sidewaystable}[htbp]
\centering
\begin{tabular}{lccccccc}
  \hline
 & \multicolumn{3}{c}{Men by Experience Class} & \multicolumn{3}{c}{Women by Experience Class} & \multirow{2}{*}{\parbox{0.1\linewidth}{All Teachers 0-9 Years}}\\ \cline{2-4} \cline{5-7}
 & 1-3 years & 4-6 years & 7-11 years & 1-3 years & 4-6 years & 7-11 years &  \\ 
  \hline
Base year salary (log) & -0.006 & 0.019 & 0.041 & -0.003 & 0.020 & 0.012 & 0.006 \\ 
   & (0.009) & (0.012) & (0.016) & (0.005) & (0.007) & (0.010) & (0.003) \\ 
  Adjusted salary (log) & 0.004 & 0.003 & 0.020 & -0.001 & 0.012 & 0.017 & 0.006 \\ 
   & (0.007) & (0.009) & (0.013) & (0.004) & (0.006) & (0.008) & (0.003) \\ 
  Percent proficient & 4.2\% & 3.0\% & 2.5\% & 6.3\% & 5.7\% & 5.3\% & 5.4\% \\ 
   & (0.7\%) & (0.8\%) & (1.0\%) & (0.4\%) & (0.5\%) & (0.6\%) & (0.2\%) \\ 
  Percent Hispanic & -1.4\% & -0.3\% & -0.2\% & -1.7\% & -1.6\% & -1.1\% & -1.4\% \\ 
   & (0.3\%) & (0.4\%) & (0.5\%) & (0.2\%) & (0.2\%) & (0.3\%) & (0.1\%) \\ 
  Percent black & -5.4\% & -2.1\% & -3.8\% & -8.6\% & -6.8\% & -6.9\% & -7.0\% \\ 
   & (1.0\%) & (1.1\%) & (1.2\%) & (0.6\%) & (0.7\%) & (0.8\%) & (0.3\%) \\ 
  Percent subsidized lunch & -7.4\% & -3.7\% & -4.4\% & -9.5\% & -7.0\% & -7.6\% & -7.9\% \\ 
   & (1.1\%) & (1.4\%) & (1.7\%) & (0.6\%) & (0.9\%) & (1.0\%) & (0.4\%) \\ 
   \hline
\end{tabular}
\caption{Average Change in Salary and District Student Characteristics (and Standard Deviations) for Teachers Changing Districts, by Gender and Experience} 
\label{tbl:change_by_ge}
\end{sidewaystable}

Table \ref{tbl:change_by_ge} replicates Table 3 of HKR, and confirms its
most important insights. Raw salary differentials predict teacher
mobility, but the average pay differential is not on average very large
-- only about \$275, or 1.5\% higher than the counterfactually expected
wage that would have obtained had the district-switching teacher
remained in their current district. This premium increases with age for
both male and female teachers.

\begin{figure}[htbp]
\centering
\includegraphics{figures/potential_premium-1.pdf}
\caption{\label{fig:premia}How Much Do Teachers Stand to Gain from
Changing Districts throughout Their Careers?}
\end{figure}

One potential explanation of the weakness of the wage results is that
there simply is not sufficient heterogeneity among available contracts
to generate mobility incentives. Figure \ref{fig:premia} demonstrates
that this is not a likely explanation. No matter their current
experience or certification level, a teacher in a district paying the
25th percentile of wages for that experience-certification cell would
gain on average 16.5695559\% by changing to a district at the 75th
percentile. Especially for younger teachers, this potential gain would
accumulate annually to become a hefty sum over the course of the career
-- discounting the average annual gain for Master's-certified teachers
at 6\% and adding over 20 years, this is roughly \$1e+05 on the table;
results are more dramatic for teachers further in the tails of the wage
distribution.

Attempting to isolate the influence of district characteristics on wage
effects, HKR suggest comparing the differential leverage of residual
wages to get a more focused estimate of the association between wages
and mobility\footnote{HKR mention they failed to adjust the standard
  errors associated with the adjusted wage differentials to account for
  the fact that they involve residuals from a regression. We explored
  accounting for this by bootstrapping the regression through resampling
  teachers and recalculating residuals, but confirm that little changes
  as a result.}. We run a similar regression using the payscales
estimated in the companion paper, but evaluate separate regressions not
just for each level of experience, but also for each certification
track. This leads to a boost in the overall fraction of explained
variance from 60\% cited by HKR to 87\% here; as in HKR, other included
covariates are consistently significant, suggesting their strong
independent correlation with salary levels.

Unlike HKR, we find the demographic-independent wage differentials to be
no more important than the uncontrolled raw wages, with the predicted
wage improvement amounting to 0.6\%. In further contrast to HKR, we find
a positive relationship between experience and residual wage
differentials, with mid-career district switchers experiencing roughly
1.3\% higher wages upon arrival to their new employer, by contrast to
the null relationship for probationary teachers. This pattern is
consistent across the dimension of certification which was ignored by
HKR, suggesting the opposite result cannot be attributed to bias
introduced by movement patterns of Bachelor's- vs.~Master's-certified
instructors.

Student demographic differentials are very important for predicting
teacher turnover, a finding which held in Texas as it does in Wisconsin.
Most distinguished in all experience classes and for both genders are
changes in measures of student performance, student poverty and the
percetnage of black students -- district switchers end up at schools
with 5\% more students at grade level overall, an effect which is
stronger for female teachers and for young teachers. They also end up on
average with about 8\% fewer students (school-wide) eligible for
subsidized lunch and 7\% fewer black students. While this finding would
need to be bolstered with experimental or quasi-experimental evidence,
it hints at the potentially limited scope of teacher labor market
policies intended to ameliorate teacher supply problems in hard-to-serve
districts -- schools can much more easily exert influence over their
compensation policies than they can dictate their student bodies, but
the latter is more efficacious (see Fulbeck
\protect\hyperlink{ref-fulbeck}{2014} and Glazerman et al.
(\protect\hyperlink{ref-glazerman}{2013})).

\begin{sidewaystable}[htbp]
\centering
\begin{tabular}{lccccccc}
  \hline
 & \multicolumn{3}{c}{Men by Experience Class} & \multicolumn{3}{c}{Women by Experience Class} & \multirow{2}{*}{\parbox{0.1\linewidth}{All Teachers 0-9 Years}}\\ \cline{2-4} \cline{5-7}
 & 1-3 years & 4-6 years & 7-11 years & 1-3 years & 4-6 years & 7-11 years &  \\ 
  \hline
Base year salary (log) & 0.017 & 0.016 & -0.011 & -0.017 & 0.016 & -0.000 & -0.001 \\ 
   & (0.014) & (0.023) & (0.036) & (0.012) & (0.014) & (0.022) & (0.007) \\ 
  Adjusted salary (log) & 0.010 & -0.026 & -0.041 & -0.005 & 0.008 & 0.013 & -0.001 \\ 
   & (0.011) & (0.017) & (0.027) & (0.010) & (0.010) & (0.015) & (0.006) \\ 
  Percent proficient & 2.1\% & 3.1\% & 1.2\% & 4.3\% & 2.9\% & 4.7\% & 3.5\% \\ 
   & (1.0\%) & (1.2\%) & (1.5\%) & (0.6\%) & (1.0\%) & (1.3\%) & (0.4\%) \\ 
  Percent Hispanic & -0.7\% & 0.3\% & -0.3\% & -1.3\% & -1.2\% & -1.1\% & -1.0\% \\ 
   & (0.5\%) & (0.6\%) & (0.8\%) & (0.3\%) & (0.4\%) & (0.6\%) & (0.2\%) \\ 
  Percent black & -1.7\% & -0.4\% & -2.1\% & -4.3\% & -2.8\% & -5.0\% & -3.3\% \\ 
   & (1.3\%) & (1.2\%) & (1.6\%) & (0.9\%) & (1.3\%) & (1.8\%) & (0.5\%) \\ 
  Percent subsidized lunch & -5.9\% & -5.6\% & -2.6\% & -7.2\% & -3.8\% & -6.1\% & -5.9\% \\ 
   & (1.6\%) & (2.0\%) & (3.0\%) & (1.0\%) & (1.5\%) & (1.9\%) & (0.6\%) \\ 
   \hline
\end{tabular}
\caption{Average Change in Salary and District Student Characteristics (and Standard Deviations) for Teachers Changing to a district more than 50 Miles Away, by Gender and Experience} 
\label{tbl:change_far_by_ge}
\end{sidewaystable}

One major aspect of teacher mobility glossed over by HKR is geographic
separation. A wide variety of frictions may be geospatially-related or
-generated -- social and professional networks tend to be concentrated
locally; there are typically substantial fixed costs involved in moving
(real estate closing fees, moving expenses, etc.); preferences may
depend on climate/geography; and so on. As a first pass at exploring how
long-distance moves may differ in nature from those over short
distances, we reproduce in Table \ref{tbl:change_far_by_ge} the analysis
of Table \ref{tbl:change_by_ge} for only those where the distance
between the origin and destination school exceeded 50 miles (a distance
deemed sufficient to likely entail a physical move rather than simply an
adjusted commute).

The preeminent distinction of long-distance moves is moderation -- all
average demographic differentials moderate towards zero, suggesting a
diminution of the importance of these aspects in this population. The
noteworthy exception to this trend is among probationary teachers --
young males experience wage increases in an uprooting move, while young
females experience further declines for long moves. More detailed data
would be needed to explore the mechanism at work behind this observation
(in particular, none of the differences -- male vs.~female or short-
vs.~long-distance moves -- have \(p\) values below .05), but one
explanation is a higher willingness among bachelors to change scenery
completely, while younger women may tend to be married and moving with
their partners. In any case, the overall importance of wages in
long-distance moves is close to zero, suggesting wage differentials are
either of secondary or tertiary concern in the associated decision
processes, or that there is unsufficient heterogeneity in wages at such
distances to generate enough moves so motivated, though the case of
young male teachers does weaken the latter explanation.

\begin{table}[htbp]
\centering
\begin{tabular}{lp{.135\textwidth}p{.135\textwidth}p{.135\textwidth}p{.135\textwidth}}
  \hline
 & \multicolumn{2}{c}{\multirow{2}{*}{\parbox{0.2\linewidth}{District Average Characteristics}}} & \multicolumn{2}{c}{\multirow{2}{*}{\parbox{0.2\linewidth}{Campus Average Characteristics}}}\\
 & & & & \\ \cline{2-5}
 & Large Urban to Suburban & Suburban to Suburban & Large Urban to Suburban & Suburban to Suburban \\
  \hline
Base year salary (log) & -0.065 & 0.016 & --- & --- \\ 
   & (0.015) & (0.007) &  &  \\ 
  Adjusted salary (log) & -0.013 & 0.010 & --- & --- \\ 
   & (0.011) & (0.006) &  &  \\ 
Average Student Characteristics & & & & \\
  \quad Percent proficient & 37.9\% & 0.9\% & 35.1\% & 0.1\% \\ 
   & (0.6\%) & (0.4\%) & (1.2\%) & (0.6\%) \\ 
  \quad Percent Hispanic & -11.3\% & -0.6\% & -7.3\% & -0.4\% \\ 
   & (0.4\%) & (0.2\%) & (1.3\%) & (0.2\%) \\ 
  \quad Percent black & -56.9\% & -0.6\% & -59.7\% & -0.5\% \\ 
   & (0.8\%) & (0.3\%) & (1.8\%) & (0.4\%) \\ 
  \quad Percent subsidized lunch & -55.7\% & -1.7\% & -61.1\% & -1.6\% \\ 
   & (1.2\%) & (0.5\%) & (1.3\%) & (0.7\%) \\ 
   \hline
\end{tabular}
\caption{Average Change in Salary and in District and Campus Student Characteristics (and Standard Deviations) for Teachers with 1-10 Years of Experience Who Change Districts, by Community Type of Origin and Destination District} 
\label{tbl:change_by_urb}
\end{table}

Table \ref{tbl:change_by_urb}, which parallels Table 4 of HKR, again
uncovers a labor market functioning similar to that in Texas. In
particular, while HKR find Large Urban - Suburban district switchers
penalize themselves in pay but are rewarded in demographic-adjusted pay,
Wisconsin teachers lose out on both measures when leaving Large Urban
districts, albeit the residual pay penalty is much lower than that of
nominal pay. This difference does not appear to be attributable to HKR's
exclusion of certification as a conditioning variable, as the pattern
here differs insignificantly by degree.

The other results of HKR are confirmed in even more dramatic fashion.
There is strong evidence of selection on the student performance metric,
which does vary quite widely in suburban districts. Teachers leaving
Milwaukee tend to end up at districts with 38\% more students deemed to
be at grade level on the state standardized test. On the other hand,
teachers leaving Large Urban districts (i.e, Milwaukee) for the suburbs
experience a precipitous drop of 57\% black students and 56\% subsidized
lunch eligibility. This is practically a tautological result, as the
student demographics outside of urban areas in Wisconsin are pretty
uniformly non-minority -- about 90\% of suburban districts have fewer
than 10\% black students, and about 60\% have fewer than 2\% black
students, whereas Milwaukee is about 60\% black. Similarly, teachers
leaving Milwaukee for the suburbs have little choice but to end up in a
district with far fewer economically disadvantaged students -- whereas
73\% of Milwaukee students are eligible, the median percentage in
suburban schools is 12\%.

This phenomenon is reflected further in the suburban-to-suburban moves,
which reflect little change in the ethnic/racial makeup of student
bodies, since a dramatic shift would demonstrate very strong influence
of this factor. We also find evidence of selection into economically
better-off districts among suburban switchers, but the magnitude of this
difference is attenuated with respect to that reported by HKR. We do not
find patterns of selection on student performance as strongly as was
found in HKR. This may be a reflection of the crudeness of the
proficiency measure as compared to the more variable raw scale score
measures used by HKR. Lastly, we confirm the finding of HKR that there
does not appear to be evidence that teachers are able to select into the
more desirable schools within their target districts -- The differences
in campus-level characteristics are almost identical to the differences
in district-level characteristics. This is likely a reflection of
supply-side constraints, as the choicest appointments in a district may
be awarded to long-serving serving teachers (promotion from within), as
well as suburban districts perhaps having only a small number of schools
at which to teach a given grade level/subject.

\begin{table}[htbp]
\centering
\begin{tabular}{lp{.1\textwidth}p{.1\textwidth}p{.1\textwidth}p{.1\textwidth}}
  \hline
 & \multicolumn{2}{c}{Between District Moves} & \multicolumn{2}{c}{Within District Moves}\\ \cline{2-5}
 & Black Teachers & Hispanic Teachers & Black Teachers & Hispanic Teachers \\
  \hline
Percent proficient & 10.7\% & 8.0\% & 2.7\% & 2.2\% \\ 
   & (3.4\%) & (5.6\%) & (0.9\%) & (1.3\%) \\ 
  Percent Hispanic & 3.2\% & -14.8\% & 1.0\% & -7.7\% \\ 
   & (1.4\%) & (7.3\%) & (0.9\%) & (2.3\%) \\ 
  Percent black & -21.1\% & -0.6\% & -2.1\% & -0.3\% \\ 
   & (5.0\%) & (5.0\%) & (1.4\%) & (2.0\%) \\ 
  Percent subsidized lunch & -19.1\% & -15.5\% & -3.5\% & -4.7\% \\ 
   & (7.7\%) & (6.7\%) & (0.8\%) & (1.3\%) \\ 
  Number of teachers & 81 & 37 & 638 & 228 \\ 
   \hline
\end{tabular}
\caption{Average Change in District and Campus Student Characteristics (and Standard Deviations) for Black and Hispanic Teachers with 1-10 Years of Experience who Change Campuses} 
\label{tbl:change_by_eth}
\end{table}

HKR examine the state of Texas, which features substantially more ethnic
heterogeneity than does Wisconsin. As a result, they are better-equipped
to identify heterogeneity in preferences by teacher ethnicity.
Wisconsin, however, only has 2,384 of its 49,755 teachers non-white, so
our results are underpowered relative to HKR. Table
\ref{tbl:change_by_eth} presents these results, which parallel HKR Table
5. Given how few observations we have of black or Hispanic teachers
switching districts, we eschew any temptation to interpret these
results. Only black switchers within districts provide enough records to
interpret meaningfully; in Wisconsin, we find (in contrast to white
within-district switchers) black teachers tend to migrate to
economically better-off and higher-performing schools (white teachers
only tend to select on the latter characteristic).

\begin{table}[htbp]
\centering
\begin{tabular}{p{.3\textwidth}p{.15\textwidth}p{.15\textwidth}p{.15\textwidth}}
  \hline
Quartile of Distribution & Probability Teachers Move to New School within District & Probability Teachers Move to New District & Probability Teachers Exit Public Schools \\ 
  \hline
Residual salary & & & \\
\quad Highest & --- & 1.4\% & 4.5\% \\ 
  \quad 3rd & --- & 1.8\% & 5.0\% \\ 
  \quad 2nd & --- & 1.9\% & 5.1\% \\ 
  \quad Lowest & --- & 1.8\% & 4.6\% \\ 
Percent proficient & & & \\
  \quad Highest & 4.6\% & 1.9\% & 4.5\% \\ 
  \quad 3rd & 4.7\% & 2.2\% & 4.5\% \\ 
  \quad 2nd & 5.2\% & 1.7\% & 4.7\% \\ 
  \quad Lowest & 6.1\% & 2.1\% & 4.8\% \\ 
Percent eligible for reduced-price lunch & & & \\
  \quad Highest & 7.1\% & 2.0\% & 5.4\% \\ 
  \quad 3rd & 5.6\% & 1.7\% & 4.0\% \\ 
  \quad 2nd & 4.2\% & 2.0\% & 4.2\% \\ 
  \quad Lowest & 3.7\% & 2.2\% & 4.7\% \\ 
Percent Black & & & \\
  \quad Highest & 7.3\% & 2.1\% & 6.1\% \\ 
  \quad 3rd & 5.0\% & 1.5\% & 4.4\% \\ 
  \quad 2nd & 4.8\% & 1.9\% & 4.2\% \\ 
  \quad Lowest & 3.5\% & 2.4\% & 3.7\% \\ 
Percent Hispanic & & & \\
  \quad Highest & 7.6\% & 1.8\% & 5.7\% \\ 
  \quad 3rd & 4.5\% & 2.0\% & 4.5\% \\ 
  \quad 2nd & 4.4\% & 2.0\% & 4.3\% \\ 
  \quad Lowest & 4.1\% & 2.2\% & 4.0\% \\ 
   \hline
\end{tabular}
\caption{School Average Transition Rates by Distribution of Residual Teacher Salary and Student Demographic Characteristics (data weighted by number of teachers in school)} 
\label{tbl:change_by_quartile}
\end{table}

To the end of examining heterogeneity in the impact of school and
district characteristic differentials on teacher mobility, HKR present
their Table 6, which breaks down the three exit rates for each
(weighted) quartile of the covariate distribution. We replicate that
analysis here in Table \ref{tbl:change_by_quartile}. Saliently, our
results for the correlation of school characteristics for
within-district movers are nearly identical to those found in Texas,
which gives a stronger indication that we have identified some
fundamental nonpecuniary mechanisms driving sorting among schools in a
district.

Differences with respect to the results in Texas begin to emerge for the
other destinations of school leavers (other districts and other
professions). As noted in Table \ref{tbl:move_by_exp}, overall rates of
switching districts are quite low compared to Texas and national
averages; conditional on this, the patterns of movement by quartile of
residual salary exhibit a similar pattern to that in Texas, with
teachers in the lowest quartile about 25\% more likely to change
districts than teachers in the highest residual pay quartile. By
contrast to HKR, however, who found the opposite association, we find
the same trend (at attenuated magnitudes) with respect to leaving
Wisconsin public schools, suggesting salary considerations are also
important for teachers considering options outside of public school
teaching (or in other states).

We also find fairly strong patterns in quitting associated with
subsidized lunch eligibility and with the ethnic makeup of schools, with
teachers at the most economically advantaged schools 7\% less likely to
exit teaching; similar numbers obtain for both the quantity of black and
of Hispanic students. For teachers moving within districts, however, we
observe the opposite pattern, which is difficult to explain \emph{prima
facie}, and suggests the presence of confounding factors distorting the
patterns away from what should be expected.

\begin{sidewaystable}
\begin{center}
\begin{tabular}{l c c c c c }
\hline
 & \multicolumn{4}{c}{Teacher Experience} \\ \cline{2-6}
 & 1-3 years & 4-6 years & 7-11 years & 12-30 years & >30 years \\
\hline
First year base salary (log)                & $0.03$        & $-0.09^{**}$  & $-0.03$       & $0.01$       & $-0.14^{**}$ \\
                                            & $(0.03)$      & $(0.03)$      & $(0.02)$      & $(0.01)$     & $(0.06)$     \\
First year base salary (log) * female       & $-0.07^{*}$   & $0.09^{**}$   & $0.02$        & $-0.02$      & $0.15^{**}$  \\
                                            & $(0.03)$      & $(0.03)$      & $(0.02)$      & $(0.01)$     & $(0.06)$     \\
Campus average student characteristics      &               &               &               &              &              \\
\quad Percent proficient                    & $-0.07$       & $0.03$        & $-0.01$       & $-0.00$      & $-0.06$      \\
                                            & $(0.05)$      & $(0.04)$      & $(0.02)$      & $(0.01)$     & $(0.06)$     \\
\quad Percent eligible for subsidized lunch & $-0.08^{*}$   & $-0.08^{**}$  & $-0.07^{***}$ & $-0.01$      & $0.08^{*}$   \\
                                            & $(0.03)$      & $(0.03)$      & $(0.02)$      & $(0.01)$     & $(0.04)$     \\
\quad Percent Black                         & $0.05$        & $0.21^{***}$  & $0.14^{***}$  & $0.09^{***}$ & $0.10$       \\
                                            & $(0.04)$      & $(0.04)$      & $(0.03)$      & $(0.02)$     & $(0.06)$     \\
\quad Percent Hispanic                      & $0.14^{*}$    & $0.17^{***}$  & $0.05$        & $-0.04^{*}$  & $-0.20^{*}$  \\
                                            & $(0.06)$      & $(0.05)$      & $(0.03)$      & $(0.02)$     & $(0.08)$     \\
Interactions                                &               &               &               &              &              \\
\quad Black * percent Black                 & $-0.21^{**}$  & $-0.13$       & $-0.03$       & $-0.03$      & $-0.12$      \\
                                            & $(0.08)$      & $(0.07)$      & $(0.05)$      & $(0.04)$     & $(0.11)$     \\
\quad Hispanic * percent Black              & $-0.19^{***}$ & $-0.19^{***}$ & $-0.12^{*}$   & $-0.12^{**}$ & $-0.02$      \\
                                            & $(0.06)$      & $(0.05)$      & $(0.05)$      & $(0.04)$     & $(0.33)$     \\
\quad Black * percent Hispanic              & $0.16$        & $-0.24$       & $-0.14$       & $0.03$       & $-0.82^{*}$  \\
                                            & $(0.25)$      & $(0.23)$      & $(0.14)$      & $(0.11)$     & $(0.39)$     \\
\quad Hispanic * percent Hispanic           & $0.16$        & $-0.16$       & $-0.20$       & $0.24$       & $0.77$       \\
                                            & $(0.28)$      & $(0.24)$      & $(0.20)$      & $(0.20)$     & $(1.05)$     \\
\hline
Observations                                & 32,998         & 30,194         & 43,581         & 99,172        & 18,487        \\
\hline
\multicolumn{6}{l}{\scriptsize{$^{***}p<0.001$, $^{**}p<0.01$, $^*p<0.05$}}
\end{tabular}
\caption{Estimated Effects of Starting Teacher Salary and Student Demographic Characteristics on the Probability that Teachers Leave School Districts, by Experience (linear probability models; Huber-White standard  errors in parentheses)}
\label{tbl:reg_lpm}
\end{center}
\end{sidewaystable}

Having identified some key patterns in moments of the data, we now move
on to try and separate the confounding effects of each of these and
other factors in affecting teacher turnover with the aim of identifying
more fundamentally the association between salient district and school
characteristics on teacher turnover. Table \ref{tbl:reg_lpm} provides
the main coefficients of interest from a simple linear probability
regression model predicting leaving a district (i.e., either switching
districts or exiting teaching); this corresponds to HKR Table 7.

By contrast to the strength of such results implied in earlier results,
the importance of student achievement has dwindled in the regression
specification, and does not come out as independently significant for
any subgroup. The same goes for base salary differentials -- in contrast
to HKR, we find little evidence of an independent influence of salary on
turnover rates, for males nor for female teachers\footnote{HKR also
  mention results not printed in their paper suggesting a paucity of
  evidence suggesting class size is an important factor in teacher
  turnover decisions; we give tepid support to this statement, as class
  size does indeed appear to be related to turnover, but somewhat weakly
  and only for younger teachers.}. This does not appear to be due to
imprecision -- the magnitude of HKR's standard errors follows closely
those found for the Wisconsin data, despite our smaller sample sizes.

HKR also found little independent evidence in favor of student economic
status factoring in to teachers' mobility decisions, but in fact this is
the source of our strongest effects. As mentioned above, it is possible
that the crude nature of the proficiency measure is only weakly
identified, and that some of the unaccounted for part of student
performance is being captured in other coefficients, especially
subsidized lunch eligibility. Even more compelling would be to associate
student performance (and other school/district-level characteristics)
more finely with the set of students actually faced by a given teacher.

The results in HKR about the differential effects of student body makeup
are largely similar to those we find in Wisconsin. White and nonwhite
teachers have opposite and significant correlations between the quantity
of minority students in their origin district and their likelihood of
leaving it. These results tend to modulate towards zero with experience,
regardless of teacher or student race category, and suggest a degree of
assortative matching on ethnicity among districts in Wisconsin (though
the patterns for whites differ sharply from those of nonwhites, the
patterns for black and Hispanic teachers are hard to distinguish).

\begin{sidewaystable}
\begin{center}
\begin{tabular}{l c c c c c }
\hline
 & \multicolumn{4}{c}{Teacher Experience} \\ \cline{2-6}
 & 1-3 years & 4-6 years & 7-11 years & 12-30 years & >30 years \\
\hline
First year base salary (log)                & $0.01$       & $-0.12^{***}$ & $-0.04^{*}$ & $0.01$       & $-0.19^{**}$ \\
                                            & $(0.03)$     & $(0.04)$      & $(0.02)$    & $(0.01)$     & $(0.06)$     \\
First year base salary (log) * female       & $-0.07^{*}$  & $0.09^{**}$   & $0.02$      & $-0.02$      & $0.17^{**}$  \\
                                            & $(0.03)$     & $(0.03)$      & $(0.02)$    & $(0.01)$     & $(0.06)$     \\
Campus average student characteristics      &              &               &             &              &              \\
\quad Percent proficient                    & $-0.15$      & $-0.03$       & $-0.08^{*}$ & $0.02$       & $0.02$       \\
                                            & $(0.08)$     & $(0.06)$      & $(0.04)$    & $(0.02)$     & $(0.10)$     \\
\quad Percent eligible for subsidized lunch & $-0.05$      & $-0.02$       & $-0.02$     & $-0.01$      & $-0.17$      \\
                                            & $(0.08)$     & $(0.06)$      & $(0.04)$    & $(0.02)$     & $(0.10)$     \\
\quad Percent Black                         & $0.25$       & $0.45$        & $0.52^{**}$ & $0.04$       & $0.11$       \\
                                            & $(0.28)$     & $(0.26)$      & $(0.18)$    & $(0.11)$     & $(0.46)$     \\
\quad Percent Hispanic                      & $0.02$       & $0.00$        & $-0.21^{*}$ & $-0.07$      & $0.06$       \\
                                            & $(0.18)$     & $(0.15)$      & $(0.09)$    & $(0.05)$     & $(0.24)$     \\
Interactions                                &              &               &             &              &              \\
\quad Black * percent Black                 & $-0.19^{*}$  & $-0.15^{*}$   & $-0.06$     & $-0.04$      & $-0.14$      \\
                                            & $(0.08)$     & $(0.07)$      & $(0.05)$    & $(0.04)$     & $(0.11)$     \\
\quad Hispanic * percent Black              & $-0.17^{**}$ & $-0.18^{***}$ & $-0.12^{*}$ & $-0.11^{**}$ & $-0.09$      \\
                                            & $(0.06)$     & $(0.05)$      & $(0.05)$    & $(0.04)$     & $(0.33)$     \\
\quad Black * percent Hispanic              & $0.05$       & $-0.16$       & $0.02$      & $0.10$       & $-0.81^{*}$  \\
                                            & $(0.26)$     & $(0.25)$      & $(0.15)$    & $(0.11)$     & $(0.41)$     \\
\quad Hispanic * percent Hispanic           & $0.14$       & $-0.10$       & $-0.12$     & $0.25$       & $0.45$       \\
                                            & $(0.28)$     & $(0.23)$      & $(0.21)$    & $(0.20)$     & $(1.07)$     \\
\hline
Observations                                & 32,998        & 30,194         & 43,581       & 99,172        & 18,487        \\
\hline
\multicolumn{6}{l}{\scriptsize{$^{***}p<0.001$, $^{**}p<0.01$, $^*p<0.05$}}
\end{tabular}
\caption{Estimated Effects of Starting Teacher Salary and Student Demographic Characteristics on the Probability that Teachers Leave School Districts with District Fixed Effects, by Experience (linear probability models; Huber-White standard errors in parentheses)}
\label{tbl:reg_lpm_fe}
\end{center}
\end{sidewaystable}

To account in a rudimentary way for district-specific hiring policies,
HKR move on to their Table 8 which repeats Table 7 with district fixed
effects. HKR note that the patterns in responsiveness to wages are the
same, though attenuated; that coefficients involving student ethnicity
are qualitatively unaffected; and that schools with high achievement
continue to exhibit lower propensities for turnover. Our results,
presented in Table \ref{tbl:reg_lpm_fe}, are similar in that they
closely resemble the results without fixed effects, but with noted
attenuation and weaker precision.

The most notable difference relative to table \ref{tbl:reg_lpm} is the
weakening of results regarding the importance of student characteristics
for white teachers. While partially attributable to a decline in
precision, this adjustment suggests much of the discovered correlation
between student characteristics and exit probability for white teachers
can be chalked up to district-to-district heterogeneity in preferences
or hiring policies.

\begin{sidewaystable}
\begin{center}
\begin{tabular}{l c c c c }
\hline
 & \multicolumn{4}{c}{Teacher Experience} \\ \cline{2-5}
 & 1-3 years & 4-6 years & 7-11 years & 12-30 years \\
\hline
I. Switch Districts                             &               &               &               &              \\
\quad First year base salary (log)           & $0.68$        & $-0.01$       & $-0.62$       & $1.37$       \\
                                                & $(0.52)$      & $(0.76)$      & $(0.71)$      & $(0.90)$     \\
\quad First year base salary (log) * female  & $-1.07^{*}$   & $-0.35$       & $-0.56$       & $-1.79^{*}$  \\
                                                & $(0.51)$      & $(0.74)$      & $(0.73)$      & $(0.89)$     \\
\quad Percent proficient                     & $-1.13$       & $0.52$        & $-0.96$       & $0.86$       \\
                                                & $(0.60)$      & $(0.81)$      & $(0.89)$      & $(1.02)$     \\
\quad Percent eligible for subsidized lunch  & $-0.54$       & $-0.82$       & $-2.04^{**}$  & $-0.81$      \\
                                                & $(0.40)$      & $(0.58)$      & $(0.64)$      & $(0.73)$     \\
\quad Percent Nonwhite                       & $0.25$        & $2.11^{***}$  & $2.83^{***}$  & $3.13^{***}$ \\
                                                & $(0.40)$      & $(0.54)$      & $(0.59)$      & $(0.66)$     \\
\quad Nonwhite * percent Nonwhite            & $-2.45^{***}$ & $-3.66^{***}$ & $-3.12^{**}$  & $0.08$       \\
                                                & $(0.56)$      & $(0.87)$      & $(1.04)$      & $(1.29)$     \\
II. Exit Teaching                               &               &               &               &              \\
\quad First year base salary (log)           & $0.32$        & $-2.39^{***}$ & $-0.86$       & $0.24$       \\
                                                & $(0.52)$      & $(0.55)$      & $(0.62)$      & $(0.44)$     \\
\quad First year base salary (log) * female  & $-0.66$       & $2.66^{***}$  & $0.87$        & $-0.48$      \\
                                                & $(0.51)$      & $(0.60)$      & $(0.64)$      & $(0.44)$     \\
\quad Percent proficient                     & $-0.20$       & $0.17$        & $0.07$        & $0.05$       \\
                                                & $(0.57)$      & $(0.67)$      & $(0.71)$      & $(0.44)$     \\
\quad Percent eligible for subsidized lunch  & $-0.71$       & $-1.72^{***}$ & $-2.31^{***}$ & $-0.58$      \\
                                                & $(0.39)$      & $(0.48)$      & $(0.52)$      & $(0.31)$     \\
\quad Percent Nonwhite                       & $1.26^{***}$  & $3.15^{***}$  & $2.41^{***}$  & $0.90^{**}$  \\
                                                & $(0.35)$      & $(0.43)$      & $(0.46)$      & $(0.28)$     \\
\quad Nonwhite * percent Nonwhite            & $-0.96^{*}$   & $-1.85^{***}$ & $-1.36^{**}$  & $-0.96^{*}$  \\
                                                & $(0.42)$      & $(0.44)$      & $(0.47)$      & $(0.38)$     \\
\hline
\multicolumn{5}{l}{\scriptsize{$^{***}p<0.001$, $^{**}p<0.01$, $^*p<0.05$}}
\end{tabular}
\caption{Multinomial Logit Estimated Effects of Teacher Salary and Student Demographic Characteristics on the Probabilities That Teachers Switch School Districts or Exit Teaching Relative to Remaining in Same District}
\label{tbl:reg_mlogit}
\end{center}
\end{sidewaystable}

Finally, the conflation of switching districts and exiting teaching may
mask important heterogeneity between these two choices. To separate
these competing exit risks, HKR construct Table 9, which gives
coefficients from a multinomial logit model with three choices -- remain
in district, switch districts, and exit teaching. We repeat that
analysis here, with the caveat that, given the sparsity in racial
variation present among Wisconsin teachers, we are unable to identify
the full model specified by HKR and mirrored above in Tables
\ref{tbl:reg_lpm} and \ref{tbl:reg_lpm_fe}. In light of this, and in
light of the apparent similarity in Wisconsin in the behavior of black
and Hispanic teachers described above, we specify the multinomial logit
model in terms of a more parsimonious coefficient set. Namely, we
distinguish between white and nonwhite teachers and white and nonwhite
students (instead of among white, black, and Hispanic students and
teachers).

We continue to see little evidence favoring the salience of wage
considerations for Wisconsin teachers; the strongest suggestions found
here point to the importance of wages for older male teachers in exiting
teaching, a result which is generally opposed to that found by HKR in
Texas, where salaries were generally important, but only for the
propensity to change districts. Also as in the regression
specifications, the prominence of student proficiency found by HKR fails
to make a notable appearance in Wisconsin.

With respect to the importance of student demographics, our results
again point to the same effects found in Texas. White teachers seem to
be spurred to change districts or exit teaching by highly black student
populations; the reverse is true of nonwhite teachers, who can be drawn
to remain in high-minority districts. Subsidized lunch eligibility's
strong effect observed in the combined specification is found here to be
concentrated more among those leaving teaching than those changing
districts.

\section{Conclusion}\label{conclusion}

That salary incentives appear to play such a limited role in driving
teacher churn is bound at first glance to be a disappointment for
policymakers. The most powerful predictors of turnover in educators in
Wisconsin are all basically beyond the control of administrators, who
have no readily-manipulated direct lever for assigning students to
schools\footnote{There is evidence (e.g., Richards
  \protect\hyperlink{ref-richards}{2014}) that catchment area
  manipulation (educational gerrymandering) is being used by some
  schools to select their student populations, but the equilibrium
  outcome of the strategic interactions of districts competing for the
  most ``desirable'' students is far from clear.}. HKR found school
quality (as measured by average standardized test performance) to be of
key importance for attracting/retaining teachers, but we found no
evidence that student proficiency (as measured by attainment levels on
standardized tests) is a factor in the turnover decision for Wisconsin
teachers. Regardless, manipulating school performance is famously
difficult\footnote{See, for example, the widespread cheating scandals on
  standardized tests by teachers in Chicago (Jacob and Levitt
  \protect\hyperlink{ref-jacob}{2003}), Atlanta, and Philadelphia as an
  example of the lengths professionals feel they need to go to effect
  change in testing outcomes.}, and is in fact the original goal
administrators often have in mind when they turn to labor market
policies in the first place, so that telling administrators they can
improve teacher retention by improving student performance amounts
essentially to circular reasoning.

The upside is that this paper is far from settling the debate about
welfare-maximizing teacher turnover policies. Limitations in our data
prevent us from associating to teachers anything but crude measures of
their productivity; measures such as experience, certification, and race
are famously poor predictors of teacher quality measures such as
value-added. We are thus unable to provide any input to the question of
whether \emph{high-quality} teachers have patterns of mobility which
resemble that of the teaching population as a whole, or whether
heterogeneity in their preferences can be used to devise appropriate
policies.

\section*{References}\label{references}
\addcontentsline{toc}{section}{References}

\hypertarget{refs}{}
\hypertarget{ref-anzia}{}
Anzia, Sarah F, and Terry M Moe. 2014. ``Collective Bargaining, Transfer
Rights, and Disadvantaged Schools.'' \emph{Educational Evaluation and
Policy Analysis} 36 (1). SAGE: 83--111.

\hypertarget{ref-ballou}{}
Ballou, Dale, and Michael Podgursky. 2002. ``Returns to Seniority Among
Public School Teachers.'' \emph{Journal of Human Resources}. University
of Wisconsin Press, 892--912.

\hypertarget{ref-biasi}{}
Biasi, Barbara. 2017. ``Unions, Salaries, and the Market for Teachers:
Evidence from Wisconsin.'' SSRN.
doi:\href{https://doi.org/http://dx.doi.org/10.2139/ssrn.2942134}{http://dx.doi.org/10.2139/ssrn.2942134}.

\hypertarget{ref-boyd2009}{}
Boyd, Donald J, Pamela L Grossman, Hamilton Lankford, Susanna Loeb, and
James Wyckoff. 2009. ``Teacher Preparation and Student Achievement.''
\emph{Educational Evaluation and Policy Analysis} 31 (4). SAGE
Publications Sage CA: Los Angeles, CA: 416--40.

\hypertarget{ref-boyd2005}{}
Boyd, Donald, Hamilton Lankford, Susanna Loeb, and James Wyckoff. 2005.
``Explaining the Short Careers of High-Achieving Teachers in Schools
with Low-Performing Students.'' \emph{The American Economic Review} 95
(2). American Economic Association: 166--71.

\hypertarget{ref-chettyI}{}
Chetty, Raj, John N Friedman, and Jonah E Rockoff. 2014a. ``Measuring
the Impacts of Teachers I: Evaluating Bias in Teacher Value-Added
Estimates.'' \emph{The American Economic Review} 104 (9). American
Economic Association: 2593--2632.

\hypertarget{ref-chettyII}{}
---------. 2014b. ``Measuring the Impacts of Teachers II: Teacher
Value-Added and Student Outcomes in Adulthood.'' \emph{The American
Economic Review} 104 (9). American Economic Association: 2633--79.

\hypertarget{ref-cohenvogel}{}
Cohen-Vogel, Lora, Li Feng, and La'Tara Osborne-Lampkin. 2013.
``Seniority Provisions in Collective Bargaining Agreements and the
`Teacher Quality Gap'.'' \emph{Educational Evaluation and Policy
Analysis} 35 (3). SAGE: 324--43.

\hypertarget{ref-croissant}{}
Croissant, Yves. 2012. ``Estimation of Multinomial Logit Models in R:
The Mlogit Package.'' \emph{R Package Version 0.2-2}.
\url{http://cran.r-project.org/web/packages/mlogit/vignettes/mlogit.pdf}.

\hypertarget{ref-dahl}{}
Dahl, David B. 2009. ``Xtable: Export Tables to Latex or Html.'' \emph{R
Package Version 1.8.2}, 1--5.

\hypertarget{ref-dolton}{}
Dolton, Peter, and Wilbert Van der Klaauw. 1999. ``The Turnover of
Teachers: A Competing Risks Explanation.'' \emph{Review of Economics and
Statistics} 81 (3). MIT Press: 543--50.

\hypertarget{ref-engel}{}
Engel, Mimi, Brian A Jacob, and F Chris Curran. 2014. ``New Evidence on
Teacher Labor Supply.'' \emph{American Educational Research Journal} 51
(1). SAGE: 36--72.

\hypertarget{ref-fulbeck}{}
Fulbeck, Eleanor S. 2014. ``Teacher Mobility and Financial Incentives: A
Descriptive Analysis of Denver's Procomp.'' \emph{Educational Evaluation
and Policy Analysis} 36 (1). SAGE: 67--82.

\hypertarget{ref-glazerman}{}
Glazerman, Steven, Ali Protik, Bing-Ru Teh, Julie Bruch, and Jeffrey
Max. 2013. ``Transfer Incentives for High-Performing Teachers: Final
Results from a Multisite Randomized Experiment.'' \emph{National Center
for Education Evaluation and Regional Assistance}. ERIC.

\hypertarget{ref-goldhaber2007}{}
Goldhaber, Dan, Betheny Gross, and Daniel Player. 2007. ``Are Public
Schools Really Losing Their Best? Assessing the Career Transitions of
Teachers and Their Implications for the Quality of the Teacher
Workforce.'' \emph{National Center for Analysis of Longitudinal Data in
Education Research}. CALDER.

\hypertarget{ref-goldhaber2015}{}
Goldhaber, Dan, Lesley Lavery, and Roddy Theobald. 2015. ``Uneven
Playing Field? Assessing the Teacher Quality Gap Between Advantaged and
Disadvantaged Students.'' \emph{Educational Researcher} 44 (5). SAGE:
293--307.

\hypertarget{ref-hanushek2010}{}
Hanushek, Eric A, and Steven G Rivkin. 2010. ``Constrained Job Matching:
Does Teacher Job Search Harm Disadvantaged Urban Schools?'' National
Bureau of Economic Research.

\hypertarget{ref-hanushek}{}
Hanushek, Eric A, John F Kain, and Steven G Rivkin. 2004. ``Why Public
Schools Lose Teachers.'' \emph{Journal of Human Resources} 39 (2).
University of Wisconsin Press: 326--54.

\hypertarget{ref-harris}{}
Harris, Douglas N, and Tim R Sass. 2011. ``Teacher Training, Teacher
Quality and Student Achievement.'' \emph{Journal of Public Economics} 95
(7). Elsevier: 798--812.

\hypertarget{ref-henningsen}{}
Henningsen, Arne, and Ott Toomet. 2011. ``MaxLik: A Package for Maximum
Likelihood Estimation in R.'' \emph{Computational Statistics} 26 (3):
443--58.
doi:\href{https://doi.org/10.1007/s00180-010-0217-1}{10.1007/s00180-010-0217-1}.

\hypertarget{ref-jacob}{}
Jacob, Brian A, and Steven D Levitt. 2003. ``Rotten Apples: An
Investigation of the Prevalence and Predictors of Teacher Cheating.''
\emph{The Quarterly Journal of Economics} 118 (3). Oxford University
Press: 843--77.

\hypertarget{ref-koski}{}
Koski, William S, and Eileen L Horng. 2007. ``Facilitating the Teacher
Quality Gap? Collective Bargaining Agreements, Teacher Hiring and
Transfer Rules, and Teacher Assignment Among Schools in California.''
\emph{Education} 2 (3). MIT Press: 262--300.

\hypertarget{ref-leifeld}{}
Leifeld, Philip. 2013. ``texreg: Conversion of Statistical Model Output
in R to LaTeX and HTML Tables.'' \emph{Journal of Statistical Software}
55 (8): 1--24. \url{http://www.jstatsoft.org/v55/i08/}.

\hypertarget{ref-litten}{}
Litten, Andrew. 2016. ``The Effects of Public Unions on Compensation:
Evidence from Wisconsin.''
\url{https://drive.google.com/file/d/0BwL-PvOgW_Ordzk5QXloZGM1d1k/view}.

\hypertarget{ref-loeb}{}
Loeb, Susanna, and Marianne E Page. 2000. ``Examining the Link Between
Teacher Wages and Student Outcomes: The Importance of Alternative Labor
Market Opportunities and Non-Pecuniary Variation.'' \emph{Review of
Economics and Statistics} 82 (3). MIT Press: 393--408.

\hypertarget{ref-moe}{}
Moe, Terry M. 2006. ``Bottom-up Structure: Collective Bargaining,
Transfer Rights, and the Plight of Disadvantaged Schools.''
\emph{Education Working Paper Archive}. ERIC.

\hypertarget{ref-murnane}{}
Murnane, Richard J, and Randall J Olsen. 1990. ``The Effects of Salaries
and Opportunity Costs on Length of Stay in Teaching: Evidence from North
Carolina.'' \emph{Journal of Human Resources}. University of Wisconsin
Press, 106--24.

\hypertarget{ref-ng}{}
Ng, Pin, and Martin Maechler. 2007. ``A Fast and Efficient
Implementation of Qualitatively Constrained Quantile Smoothing
Splines.'' \emph{Statistical Modelling} 7 (4). SAGE: 315--28.

\hypertarget{ref-dpi}{}
Public Instruction, Wisconsin Department of. 2017. ``School Staff:
Salary, Position \& Demographic Reports.''
\url{https://dpi.wi.gov/cst/data-collections/staff/published-data}.

\hypertarget{ref-r}{}
R Core Team. 2016. \emph{R: A Language and Environment for Statistical
Computing}. Vienna, Austria: R Foundation for Statistical Computing.
\url{https://www.R-project.org/}.

\hypertarget{ref-richards}{}
Richards, Meredith P. 2014. ``The Gerrymandering of School Attendance
Zones and the Segregation of Public Schools: A Geospatial Analysis.''
\emph{American Educational Research Journal} 51 (6). SAGE: 1119--57.

\hypertarget{ref-rivkin}{}
Rivkin, Steven G, Eric A Hanushek, and John F Kain. 2005. ``Teachers,
Schools, and Academic Achievement.'' \emph{Econometrica} 73 (2). Journal
of the Econometric Society: 417--58.

\hypertarget{ref-rockoff}{}
Rockoff, Jonah E. 2004. ``The Impact of Individual Teachers on Student
Achievement: Evidence from Panel Data.'' \emph{The American Economic
Review} 94 (2). JSTOR: 247--52.

\hypertarget{ref-rstudio}{}
RStudio Team. 2017. \emph{RStudio: Integrated Development Environment
for R}. Boston, MA: RStudio, Inc. \url{http://www.rstudio.com/}.

\hypertarget{ref-sable}{}
Sable, Jennifer. 2009. ``Documentation to the Nces Common Core of Data
Local Education Agency Universe Survey: School Year 2006-07 (Nces
2009-301).'' \emph{U.S. Department of Education}. Washington, DC:
National Center for Education Statistics.

\hypertarget{ref-scafidi}{}
Scafidi, Benjamin, David L Sjoquist, and Todd R Stinebrickner. 2007.
``Race, Poverty, and Teacher Mobility.'' \emph{Economics of Education
Review} 26 (2). Elsevier: 145--59.

\hypertarget{ref-stinebrickner}{}
Stinebrickner, Todd R. 2002. ``An Analysis of Occupational Change and
Departure from the Labor Force: Evidence of the Reasons That Teachers
Leave.'' \emph{Journal of Human Resources}. University of Wisconsin
Press, 192--216.

\hypertarget{ref-wayne}{}
Wayne, Andrew J, Kwang Suk Yoon, Pei Zhu, Stephanie Cronen, and Michael
S Garet. 2008. ``Experimenting with Teacher Professional Development:
Motives and Methods.'' \emph{Educational Researcher} 37 (8). SAGE:
469--79.

\hypertarget{ref-xie}{}
Xie, Yihui. 2016. \emph{Knitr: A General-Purpose Package for Dynamic
Report Generation in R}. \url{http://yihui.name/knitr/}.

\hypertarget{ref-zeileis2004}{}
Zeileis, Achim. 2004. ``Econometric Computing with Hc and Hac Covariance
Matrix Estimators.'' \emph{Journal of Statistical Software} 11 (10):
1--17. \url{http://www.jstatsoft.org/v11/i10/}.

\hypertarget{ref-zeileis2006}{}
---------. 2006. ``Object-Oriented Computation of Sandwich Estimators.''
\emph{Journal of Statistical Software} 16 (9): 1--16.
\url{http://www.jstatsoft.org/v16/i09/.}

\hypertarget{ref-zeileis2002}{}
Zeileis, Achim, and Torsten Hothorn. 2002. ``Diagnostic Checking in
Regression Relationships.'' \emph{R News} 2 (3): 7--10.
\url{https://CRAN.R-project.org/doc/Rnews/}.


\end{document}
