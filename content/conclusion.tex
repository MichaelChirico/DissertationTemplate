U.S. cities have a mixed record in their ability to collect taxes from
their residents.  Some (Boston, Charlotte, San Francisco, San Antonio)
do a good job, collecting over 98 percent of property taxes due,
others (New York, Philadelphia, St.  Louis) are less successful,
collecting in the neighborhood of 90 percent, and finally some, such as Detroit and Flint,
collect less than 70 percent of property taxes owed (Chirico, et.
al., \citeyear{CILMS-16}).  Collecting property taxes should be
straightforward; both the city and the property owner know exactly
what is due.  While scofflaws, those permanently in arrears, are a
problem, most tardy tax payments are because residents forget or
are hoping to ``let it ride'' and not be noticed.  We provide in Chapter 3 an
extension of the O'Donoghue-Rabin's \citeyear{DR-99} theory of
procrastination to explain this behavior. We then test three competing explanations
incorporated in our model using a field experiment on property tax
compliance in Philadelphia.

Our empirical analysis reached three conclusions.  First, there is
strong evidence that salience is important.  A simple reminder will improve compliance.  The
rate of compliance rose by 4 percent with a simple reminder above that
of our holdout sample that received no reminder.  But the effects of
the reminder decline over time.  There is no evidence that having
received a reminder in 2015, and having even paid your taxes, improves
your chance of compliance when paying your 2016 taxes.  These results 
strongly suggest tardy taxpayers lack salience which is consistent with Akerlof's \citeyear{Akerlof-91} work
on procrastination.  Taxpayers may have a limited capacity to remember
and process tax (and benefit) information when making their spending
and financial decisions.  An explicit reminder that brings that
information to the fore can encourage payment.  In this regard our
results are consistent with those in Chetty, Looney, and Kroft
\citeyear{chetty2009salience} on the role of saliency in the payment
of sales taxation and the results in Bhargava and Manoli
\citeyear{Bhargava-15} on the take-up rate for welfare benefits.

Second, a reminder letter with a ``message'' can improve compliance
above a simple reminder, but the content of the message matters.  Two
of our reminder letters stressed increased penalties for
non-compliance; one threatened to place a lien on the property if
taxes are not paid and the second threatened a lien and the risk of an
immediate sheriff's sale if taxes are not paid.  Both had a
significant impact on compliance, raising the rate of compliance by 9
percent over that of taxpayers who received no reminder at all.  This finding
strongly supports the theory that tardy taxpayers lack sufficient deterrence.
The messages that did not improve compliance above that of a simple
reminder letter were our four ``tax morale'' messages: one stressing
your taxes are needed for neighborhood services such as trash
collection and the local park, a second that your taxes pay for
important city-wide services such as education and protection, a third
that 9 of 10 other Philadelphians pay their taxes on time, and a
fourth that paying one's taxes is an important component of the
democratic contract. It is important to stress, however, that the impact of any nudge on behavior is conditional on the content of the message, its fiscal context, and affected taxpayers. Our results are for Philadelphia, \textit{given} its current levels of penalties, the current level of services provided by the City, and the preferences of its tardy taxpayers. Tax nudges in cities with lower penalties, better services, or more civically minded taxpayers might induce different behavioral responses. That said, the similarity of our results, both qualitatively and quantitatively, to those of Castro and Scartascini \citeyear{castro} for the property tax payments in Junin, Argentina is reassuring.

Third, the marginal impacts on city revenues of our strategies were quantitatively significant. A simple reminder letter earned the City \$28 more in additional revenues for each additional dollar of administrative cost. A reminder coupled with our most effective messages - the tax lien and sheriff letters - earned the City \$65 more in extra revenues for each dollar expended. This very high marginal revenue to cost ratio strongly suggests that well targeted nudges should be part of any City's revenue collection strategy; see Keen and Slemrod \citeyear{keen2016optimal}.

The conclusions of the second chapter are much less stark. The concept of an actively engaged classroom is one which appeals on
some level to many math educators with whom we have spoken. And in fact
many students have expressed a strong preference for the approach to
tackling new material in mathematics that we analyzed in Chapter Two. Conversely, however, many students
voiced their disdain for the active classroom and much preferred the
traditional lecture format, which in some ways offers more flexibility.
It should come as no surprise that the actively taught math classroom is
not a silver bullet for bringing struggling students to mathematical
enlightenment.

We have, however, given evidence to help confirm the active classroom's
place in the versatile educator's toolbelt. In fact, it so happened that
one of the ``traditional'' instructors (who has indeed is known pushing
the bounds of what ``traditional'' means and was one of the early
adopters of posting their recorded lectures for their students) decided
to ``activate'' their lecture hall several times, segueing from a
standard lecture to mid-sized groupwork mid-session; and their TAs were
seen to run their recitation sessions in the ``active'' groupwork
format.

In the end, the students performed slightly (though not significantly)
worse on the course final exam, but even this taken alone would not
amount to a condemnation of the utility of the active approach -- after
all, the more fundamental outcome of interest has not yet emerged, which
is the ultimate persistence rates of active-assigned students in STEM
fields.

Chapter One, too, came to less-than-monumental conclusions. That salary incentives appear to play such a limited role in driving
teacher churn is bound at first glance to be a disappointment for
policymakers. The most powerful predictors of turnover in educators in
Wisconsin are all basically beyond the control of administrators, who
have no readily-manipulated direct lever for assigning students to
schools\footnote{There is evidence (e.g., Richards
  \citeyear{richards}) that catchment area
  manipulation (educational gerrymandering) is being used by some
  schools to select their student populations, but the equilibrium
  outcome of the strategic interactions of districts competing for the
  most ``desirable'' students is far from clear.}. HKR found school
quality (as measured by average standardized test performance) to be of
key importance for attracting/retaining teachers, but we found no
evidence that student proficiency (as measured by attainment levels on
standardized tests) is a factor in the turnover decision for Wisconsin
teachers. Regardless, manipulating school performance is famously
difficult\footnote{See, for example, the widespread cheating scandals on
  standardized tests by teachers in Chicago (Jacob and Levitt
  \citeyear{jacob}), Atlanta, and Philadelphia as an
  example of the lengths professionals feel they need to go to effect
  change in testing outcomes.}, and is in fact the original goal
administrators often have in mind when they turn to labor market
policies in the first place, so that telling administrators they can
improve teacher retention by improving student performance amounts
essentially to circular reasoning.

The upside is that this chapter is far from settling the debate about
welfare-maximizing teacher turnover policies. Limitations in our data
prevent us from associating to teachers anything but crude measures of
their productivity; measures such as experience, certification, and race
are famously poor predictors of teacher quality measures such as
value-added. We are thus unable to provide any input to the question of
whether \emph{high-quality} teachers have patterns of mobility which
resemble that of the teaching population as a whole, or whether
heterogeneity in their preferences can be used to devise appropriate
policies.
