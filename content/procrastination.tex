
\section{Taxpayers As Procrastinators}

Most city residents are law abiding citizens.  If late in their city
tax payments it is unlikely it is part of a strategic plan to avoid
ever paying.  Property tax payments are computed by the city as assessed
home value times the city's property tax rate and are known both to
the city and the taxpayer.  While it is possible to avoid payment by
abandoning the property, this is very costly.  For the vast majority
of taxpayers the only issue is timely payment.  Taxpayers receive
their tax bill in January of the fiscal year with full payment or an
agreed to payment schedule required by the end of March.  Most
families have the payment withheld in an escrow account as part of
their monthly mortgage payments.  If payment, or enrollment in a
payment plan, has not been made by the end of April, the city starts
enforcement proceedings against the taxpayer.  Enforcement begins with
a reminder letter that all taxes and additional accrued interest and
penalties are now due.  In Philadelphia, those reminder letters are
mailed in early May.  We are studying the payment decisions of these
tardy, or late, taxpayers. Following the analysis of O'Donoghue
and Rabin (1999), our late taxpayer are seen as procrastinators who
struggle with the problem of when, not if, to pay their property
taxes.\footnote{We are not the first to model taxpayer compliance as a
  problem of procrastination; see Hallsworth, et. al.  (2014). We
  differ from their analysis in two ways.  First, their focus is on
  late taxpayers as possibly credit-constrained households.  That is
  less of an issue for our work as all our taxpayers are homeowners
  with assets that can used as collateral for a loan to pay taxes.  It
  is true that homeowners, particularly the elderly, may not utilize
  such loans, but that is problem of financial literacy not tax
  compliance.  Second, while we both rely upon the fundamental work of
  O'Donoghue and Rabin (1999), we amend that analysis to include the
  insight of Akerlof (1991) on the importance of ``saliency'' to the
  problem for procrastinators. We also extend the model to allow for
  active tax enforcement by the city.}

Our taxpayer makes a decision every two weeks or perhaps every month
as they pay their family bills.  They can pay their taxes today, or
postpone the decision until ``tomorrow.''  If they pay their taxes
today, they bear the immediate cost equal to the payment made.
Taxpayers enjoy a benefit from having paid their taxes, but those
benefits are not realized until ``tomorrow,'' either as the simple
relief of knowing their taxes are paid or perhaps from the good
feelings -- that is, tax ``morales'' -- of knowing they have met their
obligations to their fellow residents.\footnote{ We make the realistic
  assumption that our taxpayers will receive their public services
  whether they pay their taxes or not.  If they do not pay their
  taxes, then they will be free-riding on the good will of their more
  responsible neighbors.  }  This is O'Donoghue and Rabin's problem of
the procrastinator facing immediate costs and delayed benefits. The
decision period is today at time $t$, where $t$ represents the number
of periods since first receiving a notice that taxes are due.  In
deciding today as to whether to pay or not pay taxes, the taxpayer's
inter-temporal utility function is specified over possible dates for
payment. If the taxpayer makes a payment at time $t$, lifetime
utility at time $t$ is given by:
\begin{eqnarray}\label{eq1}
U_t^t  &=& (\varphi^{t+1} \beta \delta) \;  V - c_ t 
\end{eqnarray}
where $c_t$ is the cost of tax payment at time $t$, and $V$ is the
benefit of knowing one's taxes are paid but not enjoyed until the
period after payment.  We assume $V$ is constant for whenever taxes
are paid. 

Benefits are evaluated in today (period $t$) dollars allowing for
declining saliency to future benefits and costs at rate $\varphi$ ($0
\le \varphi \le 1$), possible present bias to all discounting at rate
$\beta$ ($0 \le \beta \le 1$), and the usual discounting of money
values at rate $\delta$ ($0 \le \delta\le 1$).  If the taxpayer plans
to make a payment at time $t + s$, then the anticipated lifetime
utility at time $t$ of that payment is given by:
\begin{eqnarray}\label{eq2}
U_t^{t+s} &=& (\varphi^{t+s+1} \beta \delta^{s+1}) \; V \; - \; (\varphi^{t+s}
\beta \delta^{s}) c_ {t+s} \; \; \; \; \; \; s=1,2, ...
\end{eqnarray}
where $c_{t+s}$ are the costs of tax payments at time $t+s$. The costs
of tax payment may rise over time with accruing interest and
penalties.

While tax payments made today are realized as a cost  ($c_{t}$) today,
tomorrow's tax payments and tomorrow's benefits are both realized in
the next period, and are, therefore, discounted for today's decisions.
Outcomes realized one period from today are discounted at the rate
$\varphi^{t+1} \beta \delta$.

In our analysis the length of each individual period is relatively
short, perhaps two weeks to a month between paying one's bills, and
the overall decision horizon of our delinquent taxpayer's is no longer
than several months.  We will, therefore, assume that $\delta=1$.  The
taxpayer may display a present bias, however, represented by a further
discounting of future costs and benefits at a rate $\beta <1$; time
consistent taxpayers do not display a present bias so $\beta=1$.
Finally, our delinquent taxpayer may be forgetful which we represent
as a declining rate of awareness or saliency, $\varphi^{t+s}$.
Constrained by bounded rationality, taxpayers may only be able to pay
attention to limited set of facts or tasks (Akerlof, 1991).  For the
forgetful taxpayer, $\varphi < 1$; for the fully aware taxpayer,
$\varphi = 1$. In the extreme future or for the very forgetful
taxpayer, $\varphi \simeq 0$ - that is, ``out of sight, out of mind.''
Introducing the concept of saliency is a relatively simply way to give
``reminders'' an explicit role in taxpayer compliance.\footnote{
  Saliency and reminders play a similar role in the behavioral
  economics of health policies; see \cite{Kessler-Zhang-14} for a
  review.} As we discuss in detail below, saliency can explain
differences in the response rates of taxpayers in the holdout sample
and taxpayers that just received a neutral reminder letter.

Our analysis focuses on the type of taxpayer who O'Donoghue and Rabin
identify as the naive procrastinator.  Here payment behavior stands in
contrast to that of the fully aware ($\varphi = 1$) and time
consistent ($\beta = 1$) taxpayer who will always pay her taxes on
time (see below) and the sophisticated procrastinator who recognizes
she is forgetful and/or present biased but is able to commit to an
optimal payment schedule in advance.  Here, that commitment device
could be an escrow account with the mortgage bank or a city arranged
tax payment plan.  In contrast, the naive procrastinator assumes that
she will remember to pay her taxes next period and do so in an
optimal, time consistent way -- but she does not.  As a result, she
may keep postponing payment until the end of the tax year when some
drastic action -- for example, court seizure of the home or
garnishment of wages -- is taken to collect all taxes, interest, and
penalties due.  Since both time consistent and sophisticated
procrastinators will have paid, or have arranged to have paid, their
property tax, they will not be in our sample of late taxpayers.  Only
naive procrastinators will be in our sample.

How does the naive procrastinator decide to pay her taxes?  She will
pay her taxes if the benefits from paying today are greater than
benefits of paying at some later date.  Following O'Donoghue and
Rabin, we assume the naive taxpayer adopts what they call a
\textit{perception-perfect strategy} and pays her taxes today only if
doing so gives them more perceived utility today than by paying at
some future date.  In our problem with constant $V$ and rising costs
$c_{t+s}$ because of accumulating interests and penalties, the best
alternative date for paying taxes will always be in the immediate next
period $ t + 1$.  If so and assuming $\delta=1$, the naive
procrastinator pays today, if the lifetime utility of paying today is
greater or equal to the lifetime utility if she delays:
\begin{eqnarray}\label{eq3}
(\varphi^{t+1} \beta) \; V - c_ t &\ge& (\varphi^{t+2} \beta) \; V -
  (\varphi^{t+1} \beta) \; c_ {t+1}
\end{eqnarray}
or if: 
\begin{eqnarray}\label{eq4}
(\varphi^{t+1} \beta) \; (V (1-\varphi) + c_{t+1}) &\ge& c_ {t}
\end{eqnarray}
The RHS of equation (\ref{eq4}) is the perceived cost of paying one's
taxes today.  The LHS of equation (\ref{eq4}) is the perceived cost of
paying taxes one period later and is equal to the actual payment of
those taxes one period later ($c_{t+1}$) plus the benefits
``forgotten'' ($V(1 - \varphi)$) because of declining saliency. With time invariant benefits ($V$), if
the perceived costs of paying one's taxes one period later are greater
than or equal to the perceived costs of paying one's taxes today, the
taxpayer will pay today.

Current period costs of compliance will equal taxes owed ($T$) plus
accumulated interest and penalties at rate $\rho$ now due from not
paying taxes in prior periods:\footnote{Strictly speaking interest and
  penalties do not begin to accumulate until some number of periods
  after the tax bill was first received. Rather than interest and
  penalties accumulating from the first date of the receipt of the tax
  bill for t periods as specified here, penalties only begin to accrue
  after a grace period.  In the case of Philadelphia, the grace period
  between when the bill is received and taxes are due is three months.
  We adopt this simpler specification for the timing of payments to
  minimize the use of superscripts for dating all the periods. All
  that is required to ensure the same level of accumulated penalties
  is to lower the rate of interest and penalties, $\rho$, in our
  specification to reflect the grace period. All comparative statics
  from the model will be the same. }
\begin{eqnarray}\label{eq5}
c_{t+s} &=& T \; (1 + \rho)^{t+s} \; \; \; s=0,1,2,...S,
\end{eqnarray}
Where $S$ is the terminal date at which point a very large penalty is
imposed upon the taxpayer for non-compliance, for example aggressive
(harassing) enforcement or seizure of one's home. In the case of
Philadelphia, after date S (December 31, 2015) the tax bill of the
non-complying taxpayer, now called a ``delinquent'' taxpayer, can be given to a collection agency and the
agency becomes the enforcer of payment. That agency can obtain a
court-order to garnish wages of the violating taxpayer. As date S
approaches the likelihood of compliance increases because of this very
large, expected penalty.

Substituting this definition into equation (\ref{eq4}) gives:  
\begin{eqnarray}\label{eq6}
\varphi^{t+1} \beta \; (V (1-\varphi) + T \; (1 + \rho)^{t+1}) &\ge&
  T \; (1 + \rho)^{t}
\end{eqnarray}
as the requirement for current period tax compliance.  More simply,
rearrange and divide both sides by $T(1 + \rho)^{t}$ and the condition
for immediate tax payment becomes:
\begin{eqnarray}\label{eq7}
\varphi^{t+1} \beta \;  (v (1-\varphi) +  (1 + \rho))  &\ge&   1
\end{eqnarray}
where $v = V/[T(1 + \rho)^{t}$] are the benefits of paying one's taxes
per dollar of taxes (and penalties) paid.  The RHS of equation
(\ref{eq7}) is the cost of paying one dollar of taxes today; the LHS
of equation (\ref{eq7}) is the perceived costs of delaying and paying
one's taxes in the next period.  The perceived costs of delay are
equal to the future benefits ``forgotten'' per dollar of taxes paid
\textit{plus} the added tax penalties from waiting.  The taxpayer will
pay her taxes today if the cost of paying a tax dollar today is less
than or equal to the costs of waiting and paying that tax dollar in
the next period.

In contrast to the naive procrastinator who is forgetful ($\varphi
<1$) and/or present biased ($\beta < 1$) and may therefore delay
payment, the fully aware ($\varphi= 1$) and time consistent ($\beta =
1$) taxpayer always pays her taxes on time -- that is, with penalties and inerest, $1 + \rho >
1$.

In addition to the usual \textit{passive} enforcement of late payments
that occurs through the payment of interest and penalties when taxes
are paid, the city may also use an \textit{activist} enforcement
strategy that audits some delinquent taxpayers at the beginning of the
current period. If audited and determined to be a delinquent taxpayer, with
probability $\pi$, the taxpayer must then pay an additional fine $F$
in the next period.  $F$ might include ``booting'' the taxpayer's car,
removing the taxpayer's children from school until payment is
received, or additional fines equal to added administrative costs plus
penalties.  A city might target its activist strategy at those
taxpayers with very large tax bills or with a year after year history of being a late taxpayer.

We assume, for simplicity, that activist enforcement is only in period
$t$ and not later.\footnote{The extension to a model in which
  enforcement occurs in each period with probability $\pi$ is not
  difficult and all results summarized in Proposition 1 also apply in
  that model.}  If the taxpayer does not pay in period $t$, then
under the activist enforcement strategy, the expected lifetime utility
in the next period if there is delay must allow for the possible
imposition of the penalty, $F$.  In this case, the expected lifetime
utility from a one period delay becomes:
\begin{eqnarray}\label{eq8}
U_t^{t+1} &=& \pi  \; [\varphi^{t+2} \beta \; V - \varphi^{t+1} \beta c_
  {t+1} - \varphi^{t+1} \beta F ) \; +  \; (1-\pi) \;  [\varphi^{t+2} \beta V -
    \varphi^{t+1} \beta c_ {t+1}], \; \; \; \mbox{or}, \nonumber  \\ 
&=& \varphi^{t+1} \beta \; [ \varphi V - c_ {t+1} - \pi F ]
 \end{eqnarray}
Now the taxpayer's decision rule is to pay if the expected utility of
delay is less than the expected utility of paying today, or with the
normalization that $f = F/(T(1 + \rho)^{t})$ , if:
\begin{eqnarray}\label{eq9}
\varphi^{t+1} \beta \; (v \; (1-\varphi) + (1 + \rho) \; + \; \pi f)
&\ge& 1
\end{eqnarray}
Note that the likelihood of making tax payments increases in the
activist enforcement parameters $\pi$ and $f$. The following proposition  
summarizes the analysis above.\footnote{Equivalently, Equation 9 can be re-written as $\varphi^{t+1} \beta (v (1-\varphi) + (1 + \rho) ) \ge 1 - \varphi^{t+1}\beta\pi f$, where $\varphi^{t+1}\beta\pi f$ can be interpreted as a ``benefit'' of early tax payment or ``forgiveness'' of tax penalties. Penalty forgiveness is a common strategy to encourage tax payment. }
\begin{prop}
Naive procrastinating taxpayers will pay their taxes today if their
perceived expected lifetime utility of delaying payment is less
than or equal to the lifetime utility of paying their taxes today, or as long as the costs from delay are greater than the costs of payment today. The likelihood of payment will increase as:
\begin{enumerate}
\item taxpayer present bias is reduced ($\beta$ rises); 
\item taxpayer saliency of future benefits and costs increases
  ($\varphi$ rises);
\item the benefits or the tax morale from the act of tax payment increases ($v$ rises);
\item the  penalties upon late payment or the subjective perception of these penalties increase
  ($\rho$ rises); and
\item activist enforcement probability ($\pi$) and the fines ($f$) increase.
\end{enumerate}
\end{prop}
Proposition 1 provides the conceptual framework for the design 
of our field experiment and the interpretation of our empirical findings. We design an experiment to evaluate the importance of three
competing theoretical explanations of non-compliance: lack of salience (Proposition 1.2), or lack of benefits or tax morale (Proposition 1.3), or lack of deterrence (Proposition 1.4). We do not test Proposition 1.1, and therefore implicitly assume that
all tardy taxpayers suffer from a common rate of present bias. Finally, our experimental design for Philadelphia does not allow us to evaluate the importance of 
activist enforcement strategies (Proposition 1.5). 

\section{A Field Experiment }


The research setting for the experiment is the City of Philadelphia
for calendar year, 2015.  Notices of property tax payments are sent on
January 1, and the full balance of taxes are due by March 31.  If
payment has not been received by that date, or the taxpayer has not
entered into a tax payment plan with the City, then taxes are
considered tardy and interest and penalties begin to accrue.  On
April 1, the City's Department of Revenue (DoR) begins contacting all
taxpayers with unpaid accounts, informing them of taxes due and
accumulated interest and penalties for late payment.  At this time,
the City will normally send two-thirds of the tardy accounts to outside collection agencies acting as co-counsel for the City. The outside collection agencies are reimbursed at the rate of six percent of all their tardy revenues collected by December 31. The remaining one-third of the tardy accounts remain with the DoR for collection. All accounts still tardy on December 31 are designated as ``delinquent'' and then assigned to new outside collection agencies. For the purposes of our experiment the City of Philadelphia agreed to delay sending tardy accounts to the collection agencies until August 15, 2015.

Our experiment was implemented with those taxpayers newly tardy on March 31, 2015. Of the 579,828 properties in the city receiving 2015 tax bills, approximately 100,000 or 17 percent were late in payment as of April 1. Of these 100,000 properties, 27,264 still owed more than \$10 as of May 15 and had not owed property taxes from prior years. Our experiment excludes all chronically delinquent taxpayers who owed taxes from prior years. Of the 21,468 tardy taxpayers, 2,429 taxpayers owned more than one property. While all 21,468 taxpayers were included in our experiment, we focus our empirical work on the 19,333 taxpayers who owned only one property.\footnote{ As a
  robustness check we repeated our empirical analysis for the full
  sample of and the results are identical those we report in Sections
  IV and V below. }  

Our experiment began with the mailing of reminder
letters in mid-June, 2015 and continued to December 31, 2015.  Of the
tardy taxpayers with a single property, 16,940 received a standard or experimental
reminder letter and 2,088 taxpayers did not receive a
reminder.  This sample of 2,088 taxpayers became our ``holdout''
sample and the basis for identifying the importance of saliency in
taxpaying behavior. To ensure that our experiment was not contaminated
by other treatments not under our control, the DoR agreed to postpone
all other enforcement activities until August 15.  In particular, the
outside collection agencies were not allowed to begin their collection
efforts until after that date.  The likely earliest date that those
efforts led to any contact with a taxpayer is September 1.

Each reminder letter was
approved by City's DoR to ensure that it could be understood by a
taxpayer with at least a fourth or fifth grade level of English
reading comprehension.  Each letter also provided contact information
for assistance for non-English speaking taxpayers.  Translation were
available for a number of different languages.\footnote{Templates of the 
``reminder only'' and ``lien'' letters  are attached in the  appendix. 
The full  template for the other letters are available as an online appendix.} 

Each reminder letter in our experiment was drafted to identify the
possible impact on taxpayer compliance of the key variables in
equation from Proposition 1.  We could not, however, measure the effect of
either taxpayer present bias ($\beta$) because our sample was limited
to tardy taxpayers only. We also cannot evaluate the direct
impact of a more activist enforcement strategy $(\pi, f)$ as the city
had not adopted such a strategy in our sample year, 2015. We can identify the potential importance of taxpayer saliency
($\varphi$), tax morales as they impact the benefits of tax payment
($v$), and interest and penalties ($\rho$). 
 For
brevity we present here the important distinguishing feature of each
letter.

\bigskip

\noindent \textit{Reminder-only}: \textbf{Our records indicate 
that you have a balance due of \textit{balance. }} If you have 
already paid, thank you.  If not, please pay now or contact us 
to arrange a payment plan.  The fastest and easiest way to pay is 
online at  www.phila.gov/pay. Paying by E-check only costs 35 cent 
-- less than the cost of a stamp!

\bigskip

 The reminder-only letter allows us to identify the potential
 importance of tax saliency to taxpayer compliance.  From Proposition 1 our holdout sample has a rate of saliency of
 $\varphi^{t+1}$ when evaluating future benefits and costs.  But those
 receiving our reminder letter today have a rate of saliency when
 evaluating future benefits and costs of $\varphi$ only.  When
 saliency is important, future taxes and benefits will be more salient
 after the receipt of the reminder, thus increasing the likelihood of
 taxpayer compliance; that is $\varphi>\varphi^{t+1}$, for $\varphi<1$. A higher rate of compliance among taxpayers
 receiving the reminder-only letter compared to those in the hold-out
 cohort identifies a separate role for saliency in taxpayer
 compliance.\footnote{Our experimental design can identify the
   presence of saliency by an increase in compliance for those
   receiving a reminder letter, but time staggered reminder letters at
   a two-week or monthly interval would be needed to identify the
   actual rate of saliency -- that is, the value of $\varphi$.  This
   was not possible within the time constraints imposed by DoR on our
   experiment.  }
   
\bigskip

\noindent \textit{Reminder plus Tax Lien}: Failure to pay your Real
Estate Taxes may result in a tax lien on your property in an amount
equal to your back taxes plus all penalties and interest.  When your
property is sold, those delinquent tax payments will be deducted from
the sale price.  By paying your taxes now, you can avoid these
penalties and interest.  Properties near you in your neighborhood that
have liens placed on them include: $<$ List Three Properties and Sale
Dates $>$ \textbf{Pay your taxes now to avoid a lien being placed on
  your property.  Our records indicate that you have a balance due of
  \textit{balance}.  }

\bigskip

\noindent \textit{Reminder plus Lien and Sheriff's Sale}: Failure to
pay your Real Estate Taxes may result in the sale of your property by
the City in order to collect back taxes.  In the past year we have
sold \textit{N} properties in your neighborhood at a Sheriff's Sale.
Included in these \textit{N} properties are the following properties
near you: $<$List Three Properties and Sale Dates$>$ \textbf{Pay your
  taxes now to prevent the sale of your property.  Our records
  indicate that you have a balance due of \textit{balance}.}

\bigskip

The reminder letter coupled with the threat of a lien, or a lien plus
a sheriff's sale of the taxpayer's home, increase the expected
interest and penalties to the costs of delay -- that is, an increase
in penalties ($\rho$).  Both letters make clear that interest and
penalties will be collected by listing neighborhood properties where
these added enforcement measures have been implemented.  A taxpayer
lien for all interest and penalties will be collected at the future
date of home sale, which may be a very large obligation if the home is
sold significantly in the future.  A lien coupled with a sheriff's
sale may occur sooner and thus have lower accumulated interest and
penalties, but the forced sale of one's home is likely to have very
high psychic costs.  Which of the two added penalties is larger, and
therefore likely to have a stronger impact on compliance, will depend
upon the circumstances of the individual tardy taxpayer.
However, both letters should increase compliance over the holdout
cohort from the reminder effect on saliency and from the added
expected penalty, and both letters should increase compliance over the
reminder-only letter from the added expected penalty.

Our final four reminder letters test for the potential role of ``tax
morale'' motives for compliance.  An appeal to a tax morale is meant
to cue a possible benefit from having paid one's taxes, apart from the
actual receipt of services those payments may make possible.  In
contrast to user fees, property tax payments are not tied to the
citizen's receipt of particular services during our experimental
period.  In effect, each delinquent taxpayer is a free rider, and the
appeal to a tax morale for payment is meant to overcome such
self-interest.  In our model of taxpayer compliance these higher
motives are captured by $v$ in Proposition 1, the morale benefits
from paying per dollar of taxes, interest and penalties paid.

We test for the importance of four such motives: 1) the value of
knowing one is a contributor to the immediate services of one's
neighborhood, $v_{N}$; 2) the value of knowing one is a contributor to
the wider services that benefit the city as a whole, $v_{C}$; 3) the
value of knowing one is part of a collective effort with other
taxpayers or ``peers'' in paying for city services, $v_{P}$; and 4)
the value of knowing one has meet one's obligations as a citizen in a
democracy, $v_{D}$.  Each of these benefits may motivate taxpayer
compliance, and our reminder letters are meant to trigger a possible
recognition of the importance of each motive.  Some tardy
taxpayers may respond to one motive, some to another, and perhaps
others to none at all if the free-rider motive is decisive.  The four
tax morale reminder letters are:

\bigskip

\noindent \textit{Reminder Plus Appeal to Neighborhood Services}: We
want to remind you that your taxes pay for essential public services
in \textit{neighborhood name}, such as $<$List Two Local Amenities$>$,
your local police officer, snow removal, street repairs, and trash
collection.  \textbf{Please pay your taxes to help the city provide
  these services in your neighborhood.} \textbf{Our records indicate
  that you have a balance due of \textit{balance}.}

\bigskip

\noindent \textit{Reminder Plus Appeal to City-Wide Services}: Your
taxes pay for important services that make a city great. Your tax
dollars are essential for ensuring all Philadelphia's children receive
a quality education and all Philadelphians feel safe in their
neighborhoods.  \textbf{Please pay your taxes as soon as you can to
  help us pay for these important services.  Our records indicate that
  you have a balance due of \textit{balance}.}

\bigskip

\noindent \textit{Reminder Plus Appeal to Peer Behavior}: You have not
paid your Real Estate Taxes.  Almost all of your neighbors pay their
fair share: 9 out of 10 Philadelphians do so.  \textbf{By failing to
  pay, you are abusing the good will of your Philadelphia neighbors.
  Our records indicate that you have a balance due of
  \textit{balance}.}

\bigskip

\noindent \textit{Reminder Plus Appeal to Civic Duty}: For democracy
to work, all citizens need to pay their fair share of taxes for
community services.  \textbf{By failing to do so, you are not meeting
  your duty as a citizen of Philadelphia.  Our records indicate that
  you have a balance due of \textit{balance}.}

\bigskip

The morale benefits from knowing one has paid one's taxes equals a
weighted average of these motivations ($v$) plus a possible additional
weight ($v_{i}$) when one of the reminder letters reinforces or
enhances the affected benefit from tax payment: $v + \sum_{i}
\omega_{i} v_{i}$, where $i =$ N, C, P, or D, and where $\omega_{i} =
1$ if a reminder letter is received targeting benefit $i$, and $v_{i}$
is the additional weight given to that motivation. We take as evidence
that an increase in tax morale increases the likelihood of tax
compliance when a tax morale reminder letter increases the rate of
compliance above that of those receiving a reminder-only letter.  If
none of the tax morale letters impact compliance above a reminder-only
letter then, at least on the margin for paying the property tax, the
free-rider motivation is decisive for tardy Philadelphia
taxpayers.  In this case, increased enforcement will need to appeal to
reminders and penalties.

  
\section{Randomization Procedure}

Randomization took place in two stages.  As a baseline control, we
randomly removed 3,000 tardy properties from the possibility of
receiving any reminder letter at all, representing 2,088 property owners.  These taxpayers (N=2,088)
became our holdout sample and allowed us to estimate the efficacy of
simply communicating with the taxpayer after the date that taxes are
due. We next grouped all remaining properties by owner and randomized all owners to treatments based on the total amount of
property taxes owed on all of their properties. 

While the vast majority of properties in the city of Philadelphia are
owned by those with just one property, approximately 10 percent of the properties are
owned by individuals or firms that own multiple properties. Since we
are interested in taxpayer compliance and not property compliance, we
identified owners of multiple properties by their legal
name and randomly assigned each owner to a treatment
group.\footnote{We lacked an objective identifier such as a social
  security.  There is some possibility that two or more different
  owners have the same name, but inspection by the authors found this
  to be very rare.  To the extent that it occurs, we consider this
  random noise to the experiment.} Any tardy taxpayer holding
multiple properties within each treatment group received the same
letter for each of those properties.  Given the high correlation
between the propensity to pay taxes and total debt owed, randomization
blocks were defined according to owner-level total debt to assure
uniformity of samples along the dimension of debt owed. Each property
assigned to receive a reminder letter was equally likely to receive
each of the seven treatments. Since most tardy property owners own only one property, our main
interest in this study will be households that only
own one property in the city. Once we restrict attention to this
sample, we have 16,940 taxpayers in the treatment group and 2,088
taxpayers in the holdout sample.  The total sample size is
19,028.\footnote{We also trimmed the sample and excluded the 28 owners
  with highest total assessed property value due to large variance in debt 
  owed among the largest delinquents. None of the findings
  reported in the chapter depend on this trimming.}  Table \ref{balance}
checks whether the treatment and holdout groups are balanced based on
the two most important variables, taxes due and assessed property
value.

Table \ref{balance} shows that randomization was successful in the
single property owner sample.  The average debt owed by each owner was \$1,287
in the treatment group and \$1,233 in the holdout sample. The average
assessed property value is \$144,145 in the treatment group and
\$142,630 in the control group. As a further test of our randomization
procedure, we also checked to see whether randomization achieved
spatial uniformity throughout the geographic expanse of the city. As
reported in Table \ref{balance} geographic balance was achieved.

Next we test whether randomization was successful among the seven
experimental treatment groups. Table \ref{balance} shows the results
for the unary owner sample. Overall, we find no evidence that would
suggest any problems with randomization. Results for multiple property
owners, which do not differ from results for unary property owners,
are reported in Table \ref{balance2} in the appendix.

\begin{sidewaystable}[htbp]
\centering
\caption{Balance on Observables (Single Property Owners)}
\label{balance}
\vspace{10mm}
\begin{tabular}{lrrrrrrrrc}
\hline
  \hline
Variable & Holdout & Reminder & Lien & Sheriff & Neighborhood & Community & Peer & Duty & $p$-value \\ 
   \hline
Amount Due (June) & \$1,233 & \$1,256 & \$1,280 & \$1,315 & \$1,289 & \$1,290 & \$1,280 & \$1,299 & 0.92 \\ 
  Assessed Property Value & \$142,630 & \$158,370 & \$130,642 & \$134,334 & \$159,079 & \$130,265 & \$130,936 & \$165,617 & 0.53 \\ 
   \hline
  Region & & & & & & & & & 0.67 \\
  \enskip Center City & 109 & 111 & 109 & 115 & 118 & 105 & 114 & 129 & \\ 
  \enskip Northeast Philadelphia & 352 & 427 & 383 & 370 & 397 & 399 & 427 & 394 &  \\ 
  \enskip North Philadelphia & 449 & 520 & 525 & 526 & 491 & 498 & 533 & 527 &  \\ 
  \enskip Northwest Philadelphia & 537 & 601 & 645 & 666 & 620 & 654 & 615 & 611 &  \\ 
  \enskip South Philadelphia & 210 & 211 & 253 & 239 & 242 & 234 & 241 & 248 &  \\ 
  \enskip West Philadelphia & 431 & 549 & 514 & 500 & 519 & 551 & 486 & 523 &  \\ 
   \hline
\# Owners & 2,088 & 2,419 & 2,429 & 2,416 & 2,387 & 2,441 & 2,416 & 2,432 &  \\ 
  \hline

\multicolumn{10}{l}{\scriptsize{$p$-values in rows 1-2 are $F$-test
    $p$-values from regressing each variable on treatment dummies. A
    $\chi^2$ test was used for the geographic distribution.}} \\
\end{tabular}
\end{sidewaystable}
\section{Empirical Results}

Table \ref{sh_lin} presents our core results for the three month
period of our experiment largely unaffected by the intervention of the
two outside collection agencies hired by the City to begin their own
enforcement efforts in September, 2015. We consider two distinct
measures of tax compliance behavior. First, did the taxpayer make any
contribution at all towards their tax bill; this is the
\textit{ever-paid} response. Second, did the taxpayer make a full
payment of their tax bill; this is the \textit{paid-in-full}
response. The sample includes only the 19,028 taxpayers who own a
single property.\footnote{We have repeated our analysis for the sample
  of taxpayers, including multi-property owning taxpayers. Results
  for the full sample are identical to those reported here for unary
  (single) property owners. We limited our reported results and
  discussion to the single property owner sample. For comparison,
  results for the sample with multiple property owners are reported in
  Appendix Tables \ref{sh_lpm_mult} and \ref{sh_logit_rob}.} For ease
of interpretation, Table \ref{sh_lin} presents OLS estimates for the
linear probability model; logit estimates are available in Tables
\ref{sh_logit} and \ref{sh_logit_rob} in the appendix and are
identical in significance and interpretation to the OLS results
reported here.

The top line of Table \ref{sh_lin} reports the mean rate of compliance
of our holdout sample for \textit{ever-paid} or \textit{paid-in-full}
one month from the starting date of the experiment (July 15) and for
the three months to the ending date of the experiment (September
15). The rate of \textit{ever-paid} compliance for taxpayers in the
holdout sample rises from 30.5 percent after one month to 51.4 percent
after three months; the rate of \textit{paid-in-full} compliance for
the holdout sample raises from 23.5 percent after one month to 40.8
percent after three months. The rising rate of compliance for the
holdout sample without receipt of a reminder letter is explained
within the \cite{DR-99} procrastination model by the presence of a
terminal date to payment (S=December 31) at which time large costs to
non-compliance can be imposed (e.g., garnishing of wages, sale of the
home, publishing of names in the Philadelphia Inquirer).

\begin{table}[htb]
\centering
\caption{Short-Term Linear Probability Model Estimates} \label{sh_lin}
\bigskip
\begin{tabular}{l c c c c }
\hline
 & \multicolumn{2}{c}{Ever Paid} & \multicolumn{2}{c}{Paid in Full} \\
          & One Month & Three Months & One Month & Three Months \\
Holdout   & $30.5$ & $51.4$ & $23.5$ & $40.8$ \\
\hline
Reminder   & $3.8^{***}$  & $3.9^{***}$  & $2.2^{*}$    & $3.0^{**}$   \\
          & $(1.4)$      & $(1.5)$      & $(1.3)$      & $(1.5)$      \\
Lien      & $9.0^{***}$  & $9.2^{***}$  & $5.6^{***}$  & $7.2^{***}$  \\
          & $(1.4)$      & $(1.5)$      & $(1.3)$      & $(1.5)$      \\
Sheriff   & $7.4^{***}$  & $8.8^{***}$  & $4.5^{***}$  & $6.8^{***}$  \\
          & $(1.4)$      & $(1.5)$      & $(1.3)$      & $(1.5)$      \\
Neighborhood & $1.7$        & $2.7^{*}$    & $-0.2$       & $1.5$        \\
          & $(1.4)$      & $(1.5)$      & $(1.3)$      & $(1.5)$      \\
Community     & $3.8^{***}$  & $2.8^{*}$    & $1.3$        & $2.5^{*}$    \\
          & $(1.4)$      & $(1.5)$      & $(1.3)$      & $(1.5)$      \\
Peer      & $3.9^{***}$  & $3.5^{**}$   & $1.8$        & $3.4^{**}$   \\
          & $(1.4)$      & $(1.5)$      & $(1.3)$      & $(1.5)$      \\          
Duty      & $2.4^{*}$    & $3.6^{**}$   & $0.7$        & $2.3$        \\
          & $(1.4)$      & $(1.5)$      & $(1.3)$      & $(1.5)$      \\
\hline
Num. obs. & 19028        & 19028        & 19028        & 19028        \\
\hline
\multicolumn{5}{l}{\scriptsize{$^{***}p<0.01$, $^{**}p<0.05$,
    $^*p<0.1$. Holdout values in levels; remaining figures relative to
    the holdout benchmark.}}
\end{tabular}
\end{table}

The next seven rows report the additional impact on compliance from
our seven treatment letters: Reminder-only, Reminder/Lien,
Reminder/Sheriff, Reminder/Neighborhood, Reminder/Community,
Reminder/Peer, and Reminder/Duty.  Receiving the reminder-only letter
increases the rate of compliance after one month for an
\textit{ever-paid} tax payment by 3.8 percent above the holdout's rate
of compliance and by 3.9 percent after three months.  Both effects are
statistically significant at the 99 percent level of confidence.
These estimates for the reminder-only letter indicate the relative
importance of saliency to taxpayer compliance behavior.  Our letter
is particularly effective early in our experiment, where the pure
effect of a reminder increases the rate of compliance after one month
by approximately 12 percent (= 3.8/30.5).  While receipt of the
reminder letter is still effective after three months, its relative
impact on compliance behavior is less, adding an additional 8 percent
(= 3.9/51.4) to the rate of \textit{ever-paid}.  The same statistical
significance and declining rate of impact on compliance is observed
for the outcome, \textit{paid-in-full}.  Here the reminder-only letter
increases the one month rate of compliance over the holdout sample by
2.2 percent on a mean rate of holdout compliance of 23.5 percent (9.4
percent improvement) and the three month rate of compliance over the
holdout sample by 3.0 percent on a mean rate of 40.8 percent (7.4
percent improvement).

Adding a message to the reminder letter has a mixed impact on tax
payer compliance.  Table 2 reports the joint effects of receiving a
reminder and a message.  Of the six messages, only the reminder/lien
and reminder/sheriff letters had a statistically robust \textit{added} impact
on compliance.  After one month, the sample receiving the
reminder/lien letter had an additional 9.0 percent rate of
\textit{ever-paid} compliance over the holdout sample's compliance
rate of 30.5 percent rate (30 percent improvement) and after three
months, an additional 9.2 percent rate of \textit{ever-paid}
compliance over the holdout sample's compliance rate of 51.4 percent
(18 percent improvement).  The impact is statistically significant at
the 99 percent level of confidence.  The results for paid-in-full
compliance for the reminder/lien letter are also quantitatively
important and statistically significant, adding 5.6 additional
compliance over the holdout sample's one month mean rate of 23.5
percent (24 percent improvement) and an additional 7.2 percent
compliance to holdout sample's three month mean compliance rate of
40.8 percent (18 percent improvement).  Comparable impacts are
observed for the sample receiving the reminder/sheriff letter, where
we observe a 24 percent (=7.4/30.5) improvement in the rate of
ever-paid compliance after one month, a 17 percent (= 8.8/51.4)
improvement in \textit{ever-paid} compliance after three months, a 19
percent (= 4.5/23.5) improvement in \textit{paid-in-full} compliance
after one month, and an 17 percent (= 6.8/40.8) improvement in
textit{paid-in-full} compliance after three months.

\begin{table}[htb]
\caption{Short-term Results: Relative to Reminder-Only}
\begin{center}
\begin{tabular}{l c c c c }
\hline
 & \multicolumn{2}{c}{Ever Paid} & \multicolumn{2}{c}{Paid in Full} \\
 & One Month & Three Months & One Month & Three Months \\
Reminder     & $34.3$ & $55.4$ & $25.8$ & $43.8$ \\
\hline
Lien         & $5.3^{***}$  & $5.3^{***}$  & $3.4^{***}$  & $4.2^{***}$  \\
             & $(1.4)$      & $(1.4)$      & $(1.3)$      & $(1.4)$      \\
Sheriff      & $3.6^{***}$  & $4.9^{***}$  & $2.3^{*}$    & $3.7^{***}$  \\
             & $(1.4)$      & $(1.4)$      & $(1.3)$      & $(1.4)$      \\
Neighborhood & $-2.1$       & $-1.2$       & $-2.5^{*}$   & $-1.5$       \\
             & $(1.4)$      & $(1.4)$      & $(1.3)$      & $(1.4)$      \\
Community    & $0.1$        & $-1.1$       & $-0.9$       & $-0.5$       \\
             & $(1.4)$      & $(1.4)$      & $(1.3)$      & $(1.4)$      \\
Peer         & $0.1$        & $-0.4$       & $-0.4$       & $0.3$        \\
             & $(1.4)$      & $(1.4)$      & $(1.3)$      & $(1.4)$      \\
Duty         & $-1.3$       & $-0.3$       & $-1.6$       & $-0.7$       \\
             & $(1.4)$      & $(1.4)$      & $(1.3)$      & $(1.4)$      \\
\hline
Num. obs.    & 16940        & 16940        & 16940        & 16940        \\
\hline
\multicolumn{5}{l}{\scriptsize{$^{***}p<0.01$, $^{**}p<0.05$, $^*p<0.1$. Reminder values in levels; remaining figures relative to this}}
\end{tabular}
\label{sh_lpm_rob}
\end{center}
\end{table}

No such consistent improvements in compliance above the reminder-only letter are observed for those
receiving a reminder letter with a ``tax morale'' message.  This is
seen most clearly in Table \ref{sh_lpm_rob} where we compare
compliance in the reminder-only sample to that of the samples
receiving one of the six message letters. In this comparison, both the
reminder/lien and the reminder/sheriff letters stressing the penalties
of noncompliance have statistically significant
and quantitatively important additional impacts on compliance above reminder-only, both
for the \textit{ever-paid} and \textit{paid-in-full} outcomes and at
the one month and three month intervals. The lien letter adds more
than a 5 percent increase in the rate of compliance above the reminder-only
letter for \textit{ever-paid} and about 4 percent to the rate of
compliance for \textit{paid-in-full}. These effects represent a 10 to
15 percent improvement in the rates of compliance over those obtained
with the reminder-only letter.  The sheriff letter also offers a
significant improvement over the reminder-only letter, though the
effects are slightly lower than those obtained with the lien letter.
Compliance rates for \textit{ever-paid} increase by 3 to 5 percent and
for \textit{paid-in-full} by to 2 to 4 percent above those achieved
with the simple reminder.  These effects represent a 9 to 11 percent
improvement in compliance performance over what had been obtained with
a reminder only. Table \ref{sh_lpm_rob} also shows most clearly the
inability of the tax morale reminders to induce greater compliance
from Philadelphia's tardy taxpayers.  Among those reminders, only the
neighborhood letter is ever statistically significant and its effect
is negative (!) for those paying in full.\footnote{\label{fn:nudges}Our results for
  both the positive impact of penalties and mixed effectiveness of tax
  morale messages are consistent with most of the current literature
  on ``nudges'' and tax compliance; see \cite{Hallsworthb-14} for a
  thorough review. In the interest of full disclosure, however, our
  pilot study \cite{CILMS-16} for this project did find a role for a
  community or duty letter in increasing compliance.  The control
  group in the pilot study received a reminder-only letter.  Three
  other groups received either a penalty letter, a community letter --
  your taxes pay for city schools, police services, and fire fighters
  -- or a combined peer/duty letter -- 9 out of 10 Philadelphians pay
  their taxes; paying your taxes is your duty. In our pilot the
  penalty letter had no additional effect on compliance over that of
  the reminder-only letter.  The community letter increased the rate
  of compliance above the reminder letter by 4 percent, but the effect
  was not quite statistically significant.  The combined peer/duty
  letter increased rate of compliance above the simple reminder letter
  by 2 percent and the effect was statistically significant at a 95
  percent level of confidence.}

It is worth speculating as to why our results here differ from those
in our pilot study.  First, the pilot was run on a much smaller sample
(3,900 single property taxpayers) and thus the results were less
precisely estimated.  Second, and more importantly, the sample for the
study included only taxpayers who had not yet paid by the middle of
November, 2014 (the time of our pilot), and thus are very close to
being what the City will classify as a ``delinquent'' taxpayer as those
who have not paid by December 31 of the tax year.  The sample
therefore consisted of the ``most-tardy'' of tardy taxpayers.  Of
these ``delinquent'' taxpayers who did make a contribution in our pilot study, the contributions were typically only partial payments of
\$50 to \$150, suggesting these households may be seriously cash
constrained.  One might then imagine that for this sample of tax
payers penalties are irrelevant; they cannot pay in full in any case.
But a morale nudge might induce some payment in the spirit of a
``charitable contribution.''  Consistent with this possible explanation
is the fact that the average rate of compliance of this sample over
the six weeks of our pilot was only 15 percent and the moral nudges
boosted the rate of those making even some contribution to no more
than 20 percent.  It would be very valuable to design a larger
experiment that seeks a compliance strategy for these very tardy or
delinquent taxpayers. For this sample, one could ``cost'' to early payment in Equation 9 to reflect a cashflow constraint there. Here the literature on liquidity constraints is relevant; see \cite{zeldes1989consumption}. Our results are similar in statistical
significance and impact to those in Castro and Scartascini's
\citeyearpar{castro} study of property tax payments in Junin Argentina,
the other major field experiment seeking to improve property tax
collection.  For Philadelphians at least, and for the residents of
Junin, it is reminders and penalties that improve compliance among
tardy taxpayers.

\begin{table}[htb]
\caption{Long-Term Linear Probability Model Estimates}
\label{ltmpme}
\begin{center}
\begin{tabular}{l c c c c }
\hline
 & \multicolumn{2}{c}{Six Months} & \multicolumn{2}{c}{Subsequent Tax Cycle} \\
 & Ever Paid & Paid in Full & Ever Paid & Paid in Full \\
Holdout      & $73.3$ & $63.2$ & $65.5$ & $52.5$ \\
\hline
Reminder     & $1.3$        & $1.5$        & $-1.4$       & $-0.7$       \\
             & $(1.3)$      & $(1.4)$      & $(1.4)$      & $(1.5)$      \\
Lien         & $3.8^{***}$  & $4.8^{***}$  & $-0.9$       & $-0.7$       \\
             & $(1.3)$      & $(1.4)$      & $(1.4)$      & $(1.5)$      \\
Sheriff      & $3.8^{***}$  & $3.0^{**}$   & $-0.6$       & $-1.1$       \\
             & $(1.3)$      & $(1.4)$      & $(1.4)$      & $(1.5)$      \\
Neighborhood & $-0.2$       & $-0.0$       & $-3.1^{**}$  & $-2.2$       \\
             & $(1.3)$      & $(1.4)$      & $(1.4)$      & $(1.5)$      \\
Community    & $0.9$        & $1.1$        & $-1.8$       & $-2.0$       \\
             & $(1.3)$      & $(1.4)$      & $(1.4)$      & $(1.5)$      \\
Peer         & $1.3$        & $2.3$        & $-1.9$       & $-1.1$       \\
             & $(1.3)$      & $(1.4)$      & $(1.4)$      & $(1.5)$      \\
Duty         & $2.1$        & $1.0$        & $-1.6$       & $-1.9$       \\
             & $(1.3)$      & $(1.4)$      & $(1.4)$      & $(1.5)$      \\
\hline
Num. obs.    & 19028        & 19028        & 19025        & 19025        \\
\hline
\multicolumn{5}{l}{\scriptsize{$^{***}p<0.01$, $^{**}p<0.05$, $^*p<0.1$. Holdout values in levels; remaining figures relative to this}}
\end{tabular}
\label{lg_pc_lin}
\end{center}
\end{table}

Table \ref{ltmpme} estimates the longer run impacts of our seven nudge
interventions on compliance.  The letters were sent on June 15th and
received soon thereafter.  The first two columns of Table \ref{ltmpme}
show the estimated effects on compliance of having received a letter
six months later, again compared to compliance behavior in our holdout
sample.  Now the reminder-only letter no longer has an impact on
compliance behavior, suggests declining saliency over time.  Reminder
letters that stress penalties from a lien or a lien plus sheriff's
sale still have influence, however.  The implied increase in expected
penalty from non-compliance appears sufficient to overcome the loss of
saliency.  But again consistent with declining saliency, the estimated
impact of the lien and sheriff letters, while still statistically
significant, are roughly half as large as their impact at the one and
three month intervals; compare Tables \ref{sh_lin} and \ref{ltmpme}.
Again, none of the tax morale nudges show a statistically significant
impact on compliance behaviors. \footnote{The six month results need
  to be interpreted with care, however, as they are no longer part of
  our experimental design.  Beginning between mid-August and
  mid-September the City allowed two private collection agencies to
  begin their efforts at collecting taxes from those in our original
  sample of 19,333 tardy taxpayers who had not yet paid their taxes,
  including those in our holdout sample.  The treatmentsâ therefore,
  become a joint intervention of our letters and the unspecified,
  proprietary strategies of the collection agencies, which we then
  compare to the collection agencies' strategies alone as they impact
  those in the holdout sample.  Whatever impact those proprietary
  strategies may have on compliance, our lien and sheriff sale letters
  still appear to have a lingering, value-added impact.}

The last two columns of Table \ref{ltmpme} carry our sample into the
next tax year, beginning with the receipt of a new property tax bill
in early January, 2016, and asks if having received a reminder letter
in June, 2015 improves compliance behavior for the payment of the 2016
taxes by June of 2016.  Consistent with the importance of saliency,
none of the 2015 reminder letters appear to have ``staying power'' into
the next tax year.  Tardy Philadelphians need constant reminders.

\section{Discussion}

While of interest as a specification and test of a behavioral theory
of tax compliance, our results are also relevant for city tax
collection policies.  As a strategy for improving collection from
tardy taxpayers, our analysis informs two important policy issues.
First, cities need revenues: Do reminders improve collection, and then
do reminders with a message raise more money than a simple reminder?
Second, in light of the recent municipal fiscal crises and the
potential for an unraveling of citizen commitment to local governance:
Do reminders with a message, and then which message, improve tax
collection as a ``nudge'' to citizen engagement? Table \ref{sh_rev}
provides answers to these two questions.


\begin{table}[htb]
\caption{Three Month Impact of Collection ``Nudges''*} 
\label{sh_rev}
\centering
\begin{adjustbox}{width=\textwidth,totalheight=.9\textheight,keepaspectratio}
\begin{tabular}{lcccccc}
  \hline
Treatment & Sample & Total Taxes & New  & Revenue/ & New  & New \% of Taxes 
\\ 
& Size & Owed & Payers & Letters & Revenues & Paid\\
\hline
Reminder & 2,419 & \$3.038 M & 95 & \$28.79 & \$69,643 & .023\\ 
  Lien & 2,429 & \$3.109 M & 224 & \$67.67 & \$164,370 & .023 \\ 
  Sheriff & 2,416 & \$3.177 M & 213 & \$64.90 & \$156,798 & .049 \\ 
Neighborhood & 2,387 & \$3.077 M & 65 & \$19.77 & \$47,191 & .015 \\ 
  Community & 2,441 & \$3.149 M & 68 & \$20.91 & \$51,041 & .016\\ 
  Peer & 2,416 & \$3.092 M & 85 & \$25.65 & \$61,970 & .020\\ 
  Duty & 2,432 & \$3.159 M & 88 & \$26.62 & \$64,739 & .020\\ 
   \hline
   \hline
  Totals & 16,490 & \$22.143 M & 838 & - & \$615,752 & .028\\
  \hline
\multicolumn{7}{p{1\textwidth}}{\scriptsize* Sample Size are the
  number of single property taxpayers in the treatment group.  Total
  Taxes Owed is the total taxes owed by single property taxpayers in
  the treatment group. New Payers equals the new payers after three
  months computed as the estimated increase in rate of compliance of
  those receiving the letter over those in the holdout sample as
  reported in Table 2; for example, for the reminder letter the number
  of new payers equals 95 = .039 x 2,419.  Revenue per letter for each
  treatment equals the median new revenue collected from those who
  received a treatment letter and made some payment (=\$ 738/letter)
  times the three month increase in compliance from each treatment
  letter; for example for the reminder letter the median estimated
  revenue per letter equals \$28.79 = .039x\$738.  New revenues for
  each treatment equals the revenue/letter times the number of single
  owner properties receiving a treatment letter: for example, for the
  reminder letter the estimated total new revenues equals \$69,643 =
  \$28.79x2,419.  New \% of Taxes Paid equals New Revenues Divided by
  Total Taxes Owed; for example, for the reminder letter .023 =
  \$69,643/\$3,038,000.}
\end{tabular}
\end{adjustbox}
\end{table}

Listed in Table \ref{sh_rev} are our seven treatments, the sample size
to which each treatment applied and total taxes owed, and then
estimates of the impact of each treatment on the number new payers
three months after receipt of the treatment letter, the average new
revenue received per letter sent, total new revenues collected from
each treatment letter above that paid by the holdout sample, and
finally, the percent of owed taxes paid because of each treatment.

For single property owners, the total number of new taxpayers above
the holdout sample from all reminder letters is 838, an average
increase in the overall rate of compliance from receiving one our
treatment letters of 4.9 percent (838/16,940).  Table \ref{sh_rev}
also provides an estimate of additional revenues raised by each of our
treatment letters and then the total revenue raised from each
treatment group.  From the perspective of the City's Department of
Revenue, our experiment was a good investment of Department resources.
Each letter cost about \$1 to process and send.  Thus estimated
benefit to cost ratios for the seven treatments ranged from a low of
\$19.77 (the Neighborhood letter) to a high of \$67.67 (the Lien
letter).  The approximately \$17,000 spent on our experiment to mail
the 16,940 treatment letters raised \$615,752 in additional city
revenues: an average benefit to cost ratio of 36.3.

Among our seven treatments, our experimental results clearly show the
power of the lien and sheriff letters compared to a simple reminder or
the tax morale nudges.  The number of new taxpayers above the holdout
sample is three to four times larger and the revenue/letter is two to
three times larger with the letters stressing penalties.  As a
consequence, total new revenues (above the holdout sample) from the
penalty letters and new revenues as a share of all taxes owed are
three to four times larger as well.  If we had sent only the lien or
sheriff's letter to the 16,940 taxpayers in our treatment groups we
would have raised \$1.15 million in new revenues rather than \$616,752
-- nearly twice as much.  The paid share of taxes owed would have
risen from our experiment's average of .028 to lien letter only of
.053.

While the seven treatments are effective on the margin and the penalty
letters particularly so, the final column makes clear that at least in
Philadelphia, our treatments will not completely solve the larger
problem of unpaid City property taxes.  The treatments encourage a 3
to 9 percent higher rate of compliance above the holdout sample, and
the typical new taxpayer pays on average about 60 percent of what they
owe. \footnote{The median taxpayer in our sample who pays taxes,
  pays \$738 towards the (average) tax bill of about \$1200, or 60
  percent.}  Thus the contribution towards total taxes owed will range
from a low of 1.5 percent for the neighborhood letter to a maximum of
5.3 percent for the lien letter.  Nudges help, and money is money, but
at least in Philadelphia, they alone will only partially solve the large problem
of tardy and then delinquent tax payments.

Money may not be all that matters with tax collection, however.
Voluntarily paying one's taxes on time is a signal that one believes
in what government is trying to do; see Posner
\citeyearpar{Posner-00}. From the
U.S. Colonies' resistance to British taxation in the 1760's to the
boycotts of the apartheid government's imposition of utility taxes on the residents of Soweto in the 1980's, refusing to pay one's
taxes is a rejection of government's performance.  In signaling games where there is a cost to
non-compliance, the more who indicate they favor your contrarian
position, the more likely you are to publicly express that position
too; see Lohmann \citeyearpar{lohmann1994dynamics} and Benabou and Tirole
\citeyearpar{benabou2011laws}.  In our case, what may have once been a
strong tax compliance outcome can unravel to a new, non-compliance
equilibrium when government no longer performs as needed for a majority of citizens; see Besley, Jensen, and Persson
\citeyearpar{besley2015norms}.  Recently, such an unraveling towards a
low compliance equilibrium can be observed in Detroit. The city's rate
taxpayer compliance for property tax collections fell from a ten year
average of .90 from 2000-2010 to a compliance rate of .68 by 2014
(Chirico, et. al., 2015).  In 2013, 47 percent of
Detroit's properties were classified as delinquent.\footnote{See Reese
  and Sands \citeyearpar{reese2013no} who conclude from their review of
  the economic and political events leading to the Detroit fiscal
  crisis that ``it is not surprising that many view the social
  contract between property taxpayers and city government as broken.''
  (p. 9) Another example of this can be seen in the 1990 taxpayer
  revolt to Prime Minister Thatcher's introduction of a local poll
  (head) tax; see Besley, Jensen and Persson
  \citeyearpar{besley2015norms}.  The regressive poll tax replaced a
  proportional property tax.  In response to widespread citizen
  resistance the poll tax was removed two years later and the property
  tax restored.  But compliance rates for the restored property tax
  were 14 percent lower than before: .83 vs. .97.  Efforts to restore
  compliance since then have stressed high penalties but it has taken
  nearly eighteen years to return to the original rates of payment.
  Expected penalties perhaps are no substitute for good governance for
  ensuring voluntary taxpayer compliance.} While nudges help, a high initial value of \textit{V} reflecting government benefits significantly greater than tax costs may be the most important determinant of the aggregate rate of taxpayer compliance and commitment to city government; see Haughwout, Inman, Craig and Luce \citeyearpar{haughwout2004local}.

