This work is dedicated to understanding and applying a wide range of topics in public economics. In three chapters, I explore public elementary and high school teachers' revealed preferences for pecuniary and non-pecuniary aspects of compensation; the potential for novel technology-assisted pedagogical modes to strengthen the STEM pipeline at American undergraduate institutions; and the efficacy of low-cost, behaviorally-founded tools for local governments to recoup their debts quickly. 

The first chapter is all about investigating the labor market for teachers and which incentives can be targeted by policymakers to improve student outcomes. Good teachers have large impacts on student achievement\footnote{See,
  e.g., Rockoff (\citeyearpar{rockoff}).}. It is
therefore imperative for public schools to be able to attract and retain
high-quality teachers. Of preeminent concern for policymakers, then, is
the strength of the various manipulable levers at their disposal for
influencing teacher labor markets. More specifically, state and local
education administrators would like to identify the policy implications
of various tools on three types of teacher mobility: intra-district
switching, where due to the collectively bargained nature of most
teachers' salaries, only nonpecuniary considerations matter,
inter-district switching, where teachers move to another school district
in the same state, and exo-district switching, where teachers leave the
public teaching workforce entirely\footnote{Policies that affect the
  supply and quality of new teachers to the profession may also be of
  considerable importance to replenishing and improving the stock of
  teachers over time, but I do not consider these channels in this
  work. See Harris and Sass (\citeyearpar{harris}),
  Wayne et al. (\citeyearpar{wayne}), and Boyd et al.
  (\citeyearpar{boyd2009}).}.

The brunt of this chapter is a replication in a new
context (Wisconsin) and time horizon (2000 - 2010) of Hanushek,
Kain, and Rivkin (\citeyearpar{hanushek}), who analyze various predictors of teacher churn. The headline results of Hanushek, Kain, and Rivkin (\citeyearpar{hanushek})
were that ``teacher mobility is much more strongly related to
characteristics of the students, particularly race and achievement, than
to salary, although salary exerts a modest impact once compensating
differentials are taken into account.'' I confirm the pith of this
conclusion, namely that student characteristics are a much better
predictor of turnover than are wage differentials, though I come to
different conclusions regarding more specific points. To wit, while I
do find strong evidence that the socioeconomic makeup of a teacher's
district predicts turnover (and that there is heterogeneity in this
effect by teacher race), the evidence I find for the importance of
wages and student achievement is far from compelling.

I explore to the extent possible potential contributors to this
discrepancy in results; most salient are the measurable differences
between Texas, where Hanushek, Kain, and Rivkin
(\citeyearpar{hanushek}) conduct their study, and
Wisconsin. Wisconsin is a largely rural state -- its largest
city/metropolitan area, Milwaukee, currently has roughly 600,000
residents (1,500,000 including the metropolitan area), making it around
the 30th-largest city in the United States\footnote{In fact, Milwaukee
  is the only city in Wisconsin considered to be ``large'' for NCES
  reporting purposes.}. By contrast, Texas has six cities larger than
this, with El Paso (\#6 in Texas) being the nearest in size to
Milwaukee. Though the non-urban parts of Texas are themselves sparsely
populated and distinctly rural, the more uniform lack of major
population centers in Wisconsin is likely to be reflected when
considerably different preferences among local vis-à-vis urban residents
for various aspects of potential teaching positions are aggregated.

To the end of exploring the pecuniary aspects of teacher turnover, I
start by expanding upon the efforts of Hanushek, Kain, and Rivkin
(\citeyearpar{hanushek}) to infer teachers' tenure-wage
paths from teacher-level data on pay. Most unionized teachers are paid
according to a salary schedule (specifying wages as an increasing
function of tenure and certification) explicated in contracts
collectively bargained at the district level. With this easily-obtained
information in hand, teachers are able to infer their future potential
wage trajectories at their own and other potential district employers.
Lacking the physical contract faced by the teachers, an econometrician
armed only with administrative data reporting actual wages in a given
year must use some imputation techniques to deduce the underlying wage
structure. I explore the utility of natural Constrained B-Splines
(COBS) to this end. COBS are an enhanced version of the traditional
semiparametric splines technique enhanced by the ability to impose a
monotonicity constraint on the resultant curve which allows the fit to
incorporate more local information from nearby experience cells.

The fidelity and utility of the resulting fitted contract curves are
supreme. In both large and small districts, COBS produces a plausible
tenure-wage arc which enables us to examine counterfactual wage levels
for mobile teachers. By comparing the fit to a small number of wage
tables obtained from actual contracts, I also learn that using COBS may
be preferable to an attempt to use actual wage tables, as the
data-derived curves can reveal latent progress of teachers towards
further certification, an aspect which is commonly observed in salary
tables but rarely included in teacher-level data.

The second Chapter continues to have improving student-teacher interactions as a motivation, but proceeds to study an older segment of the education sector -- college. With the rise of technology in the classroom have come a variety of
approaches to teaching course material based on methods unavailable in
the past due to absent technological tools. While the literature
formally aiming to identify causal effects of these new methods on
student outcomes is growing, there is still as yet no broad conclusions
about if, when, and how such new methods can be used to serve the needs
of students and/or simplify the process of learning for students and
instructors alike.

This chapter seeks to advance our understanding of factors that influence
students' engagement and learning in college mathematics. The particular
focus of the study is a comparison of traditional and active
instructional methods (with online components) in the context of an
intermediate calculus course at a mid-sized private university in the
Northeast. In theory, the appeal of active learning classrooms is
that students taught in this mode will more fully engage with the
curriculum and, as a result, that they will attain deeper understanding
and mastery of mathematical concepts, thereafter proceeding to pursue
degrees in math or the sciences

I test this hypothesis by block-randomizing students between the two
pedagogical modes and monitoring their performance in the course. I
supplement the quantitative measures of student progress with
qualitative data obtained through formal course observations designed to
elicit an understanding of the differences in learning environments and
treatment fidelity among the six lecture sections.

Given substantial noncompliance observed in the data, I use the
potential outcomes framework of, e.g., Rubin
(\citeyearpar{rubin}), G. W. Imbens and Rubin
(\citeyearpar{imbens}) and Angrist and Imbens
(\citeyearpar{angrist}) to construct intent-to-treat
(ITT) and local average treatment effect (LATE) estimates of active
learning environments on student performance. I incorporate the blocked
nature of randomization to our estimates by block-bootstrapping standard
errors for these estimates. I find suggestively negative point
estimates in two of the time slots, but no results are statistically
significant.

The substantial noncompliance observed in the data is suggestive of
several important considerations for randomized trial design in similar
settings. It is apparent that instructor fixed effects can be
substantial -- in our setting, there was stark contrast in the level of
experience of the traditional vis-a-vis the active learning instructors
in precisely the two sections that attracted the most non-compliers
(i.e., this level of experience is also accompanied by renown among
students). Absent randomizing instructors to pedagogical modes or
restricting the ability of students to change sections (both impossible,
practically speaking), the solution to overcoming this threat to
identification is replication -- namely, to continue to monitor the
performance of students in the two instructional modes. In the longer
run, sample size increase and balancing of fixed effects as instructors
accustom to the new method and students gain comfort with how to perform
in such a class stand to strengthen our understanding of the impacts of
active learning for student achievement.

Our experience with treatment infidelity in the pilot round of our collaborative study with the City of Philadelphia led us to design a more ambitious and tightly-controlled second iteration of our real estate tax experiment, which is the topic of Chapter 3\footnote{This chapter is a co-authored work, with Charles Loeffler, John MacDonald, Holger Sieg, and Robert Inman}. Property taxation is the primary tax for most U.S. cities.  In fiscal
year 2013, 30 percent of all local government revenues and over 73
percent of local taxes came from the property tax
\cite{barnett2013state}.  Yet collection of the tax has, in many
cities, been problematic.  While some U.S. cities do an excellent job
in collecting the tax, receiving over 95 percent of assessed revenues
in the year the tax is due, other cities have over the last ten years
done significantly worse -- notably Flint (78\%), Cleveland (84\%),
Pittsburgh (86\%), Milwaukee (87\%), Philadelphia (88\%), Detroit
(89\%), and St. Louis (89\%).\footnote{For more details, see
  \cite{CILMS-16}.}  While Flint, Detroit, Cleveland and Milwaukee
are relatively poor cities, Philadelphia and Pittsburgh are not.
Among the list of cities with outstanding tax collection records are
Buffalo, Birmingham, Houston, and New Orleans.  While city poverty is
important, it cannot be the whole explanation for low rates of
collection.  Poor tax administration is likely to be an important
contributing factor as well.

This failure to collect the property tax on time creates budget
uncertainty at best and budget deficits at worst.  Yet collecting the
property tax should be straightforward.  In contrast to collecting
self-reported taxes such as those on income, profits, and sales,
property tax obligations equal to the city's assigned assessed value
of the taxed property times the city chosen tax rate are known by both
the city and the taxpayer.  There is no uncertainty as to what is due,
or when.\footnote{Much of the current literature on tax compliance has
  focused on taxpayers truthful reporting of income or sales under the
  threat of a tax audit; see \cite{Slemrod-07} for a review and more
  recently the research of \cite{Kleven-11} and \cite{Pomeranz-15}.}
Payment is primarily a matter of enforcement.  The most common
enforcement strategy is the economic stick: fines and penalties.
Failure to pay property taxes in time leads to interest penalties
sufficiently large that there is no arbitrage advantage to waiting,
and perhaps to a significant late fine as well.

When a delinquent taxpayer does not respond to penalties and fines,
the city can take out a tax lien on the property.  A lien does not
impose any immediate direct, tangible costs on a  taxpayer
since payments are typically only realized at the time of a
transaction.\footnote{A city can also sell tax liens to investors to
  speed up the revenue collection process. Liens often sell at above
  par prices because of the foreclosure option. But selling liens to
  ``vulture investors'' can be politically costly for a city
  administration.}  However, obtaining a tax lien enables the owner of
the lien to eventually start a foreclosure process. When the owner of
a property located in a city fails to make a payment arrangement on
municipal tax levied on his or her property, that property may be
sold at auction to allow the city to collect on that unpaid debt.
However, the foreclosure process is costly and time
intensive.\footnote{Auctions are administered in Philadelphia by the
  Office of the Sheriff.  This process of offloading a property at
  Sheriff's Sale can take nine months to a year.}  While there are
some problems with the effectiveness of the existing enforcement
mechanisms, it is only possible to avoid payment by abandoning the
property in the long run. Needless to say, this is a very costly
option for most owners.

Despite the fact that there are no obvious financial gains to not
paying property taxes, we observe that a significant fraction of tax
payers do not pay on time. To explain the behavior of these
procrastinators, researchers have started to explore the effectiveness
of softer, nudge approaches or notification strategies to reinforce
the different motivations of tax compliance.  This chapter uses a field
experiment involving over 19,000 delinquent Philadelphia taxpayers to
examine the effectiveness of seven alternative strategies for
improving city property tax collection. Each involves a randomly
assigned tax ``nudge'' of a tardy taxpayer.  The first is a
simple reminder that the payment is late.  The next two involve the
reminder plus a threat of a significant sanction if payment is not
received by the end of the calendar year: a lien on the home when sold
equal to taxes due plus accrued interest and penalties \textit{or} the
lien coupled with an immediate sheriff's sale of the home to collect
the lien.  The final four nudges include the reminder coupled with an
appeal to what the tax compliance literature has called a ``tax
morale'' motive for paying one's taxes.\footnote{See
  \cite{Luttmer-14} for a review of the tax morale strategies for tax
  compliance.  They identify three tax morale motivations in the
  literature, each grounded in a positive gain in utility from the act
  of paying one's taxes.  These include: 1) a motive from reciprocity
  where the taxpayer recognizes they are part of a larger group
  playing a non-cooperative game with other taxpayers for the provision of public goods; 2) a motive
  from peer behavior where the taxpayer gains utility from knowledge
  that they are part of larger group of contributors; and 3) an
  intrinsic motivation that provides a direct utility benefit from the
  act of paying one's taxes.  \cite{Luttmer-14} also mention taxpayer
  culture and taxpayer behavior other than utility maximization as
  additional explanations for the rate of taxpayer compliance. }  The
four morale motives included here are: first, a reminder that taxes
pay for neighborhood services such as street repairs, trash pick-up
and the local park; second, a reminder that taxes pay for important
city-wide services such as police protection and public schools;
third, a reminder that 9 out of 10 Philadelphians have paid their
taxes and you have not; and fourth, a reminder that paying one's taxes
is an important obligation of citizenship in a democracy.  Tax
compliance after receiving one of the seven ``nudges'' is then
compared to compliance for those who have not received a ``nudge'' 
dues to random assignment to a holdout sample.


To understand the potential influence of each nudge, we model tax
delinquency as a problem of taxpayer procrastination following
\cite{Akerlof-91} and \cite{DR-99}.  Procrastination occurs 
because of present bias as in \cite{DR-99} and declining saliency as in
\cite{Akerlof-91}.  Present bias is always present.  Saliency can be
nudged by a reminder letter.  The reminder letters stressing liens or
liens plus the sale of one's home add a future expected cost to
non-payment as in the tax compliance model of
\cite{Allingham-Sandmo-72}.  The reminder letters stressing the tax
morale are modeled as utility gains to the procrastinator from paying
ones taxes.  Economic theory, therefore, plays a central role in the design and implementation
of our field experiment. Late payments arise in our model 
due to lack of salience, lack of deterrence or lack of tax morale.  
We have designed the field experiment to test the importance of these 
three competing theories. We show that the treatment effects that are identified
by our experiment have a clear interpretation in the context of the
parameters of our model.  Our experiment is, therefore, designed to explicitly
test competing models of behavior as recommended by Levitt and List
\citeyearpar{levitt2007laboratory} and \cite{cdvm-11}.

Our work here is closely related to the recent work of
\cite{Hallsworth-14} studying the effect of taxpayer nudges on the
timeliness of income tax payments in the UK and to Castro \& Scartascini's \citeyearpar{castro} study
of local property tax payments in Argentina.  Like our study, the
amount owed to the tax authorities in these two studies is known by
the authority and the taxpayer with certainty; the only issue is
payment. As here, the empirical analysis of \cite{Hallsworth-14}
follows from a model of taxpayer procrastination.  The primary focus
of their field experiments is the \textit{framing} of the morale
nudge, comparing the effectiveness of what they call a
\textit{descriptive} message (``a majority of citizens pay their
taxes'') to that of an \textit{injunctive} message (``you
\textit{should} pay your taxes because'').  Our analysis also includes
a descriptive message (``9 out 10 taxpayers have paid their tax'') and
an injunctive message (paying one's taxes is a duty of citizenship'').
We differ from the \cite{Hallsworth-14}, by including a more strongly
worded message on the penalties for non-compliance and by allowing a
longer period of study for compliance behavior (3 weeks vs. 6 months
in our study).  The longer period allows a sharper identification of
the saliency of each nudge.  Finally, they study compliance for the
payment of an important national tax; we study compliance for an
important local tax.  Like the work
here, \cite{castro} study citizen payment of their local property
taxes. They also examine effectiveness of separate nudges that stress
legal and financial consequences of non-payment, the advantage of
payment for the provision of neighborhood services (street lighting),
and the fact that seven of ten taxpayers do pay their bills on
time. However, they do not consider the effects of saliency nudges
independent of the content of the nudges, which is one of the key
objectives of our analysis.\footnote{We conducted an earlier pilot study of
  property tax compliance in Philadelphia.  The results are reported
  in \citealt{CILMS-16}. In contrast to our results here, we find evidence that  tax morale motives drive by public good provision, peer effects, and civic duty can positively impact
  property tax payment compliance; see footnote \ref{fn:nudges} below.}

Our experiment supports three central conclusions. First, saliency of
the tax obligation matters. A simple reminder letter has both a
statistically significant and quantitatively significant impact on the
rate of taxpayer compliance, though the effect wears off over
time. Second, beyond the simple reminder, the content of the ``nudge''
matters as well with those stressing rising financial penalties having
the greatest impact on compliance. Those appealing to a tax morale --
neighborhood, community, peer behavior, and civic duty -- were no more
successful than the simple reminder letter in inducing additional tax
compliance. Third, the marginal revenue benefit of our most effective message is significant, raising \$36 in new revenue for each \$1 of administrative costs. That said, however, the aggregate effect of nudges on uncollected revenue is modest, bringing in only 5\% of all revenue still owed, at least in Philadelphia.
